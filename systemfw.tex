
\section{Universal types and type operators: System $F_{\omega}$}
\begin{frame}
  We would like to be able to create a type $\text{Pair}~\text{Int}~\text{Bool}$ without
  having to write support for pair types as part of our system.
\end{frame}

\begin{frame}
  This looks a little like a function taking two types and return a type.
\end{frame}

\begin{frame}
  These kind of things are referred to as {\it type operators}.
\end{frame}

\begin{frame}
  \begin{mdframed}[frametitle={Types and kinds}]
\begin{displaymath}
    \begin{aligned}
T \quad:=\quad& ~ \ldots &\\
  & ~ X \quad\quad &type~variable\\
  & ~ T \rightarrow T \quad\quad &function~arrow\\
  & ~ \forall~X \highlight{{::} K}. T \quad\quad &univeral type\\
  & ~ \highlight{\lambda~X {::} K . T} \quad\quad &operator~abstraction\\
  & ~ \highlight{T ~ T} \quad\quad &operator~application\\
\highlight{K} \quad:=\quad& ~ &\\
   & ~ * \quad\quad &kind~of~proper~types\\
   & ~ K \Rightarrow K \quad\quad &kind~of~type~operators
    \end{aligned}
\end{displaymath}
\end{mdframed}
\medskip

\begin{overprint}
\onslide<1>
We add operator abstraction to our set of types.
\onslide<2>
That presents a need for operator application in order to use it.
\onslide<3>
We add in a {\it kind} system in order to check that our types make sense.
\onslide<4>
$\text{Pair}~\text{Int}~\text{Bool}$ makes sense.
\onslide<5>
It has kind $* \Rightarrow * \Rightarrow *$
\onslide<6>
$\text{Int}~\text{Bool}$ makes no sense at all.
\onslide<7>
It should be ill-kinded and hence ruled out.
\end{overprint}

\end{frame}

\begin{frame}
  \begin{mdframed}[frametitle={Terms and values}]
\begin{displaymath}
    \begin{aligned}
t \quad:=\quad& ~ \ldots &\\
  & ~ x \quad\quad &variable\\
  & ~ \lambda~x {:} T . t \quad\quad &abstraction\\
  & ~ t ~ t \quad\quad &function~application\\
  & ~ \lambda~X \highlight{{::} K}. t \quad\quad &type~abstraction\\
  & ~ t~\left[T\right] \quad\quad &type~application\\
v \quad:=\quad& ~ \ldots &\\
  & ~ \lambda~x {:} T . t \quad\quad &abstraction\\
  & ~ \lambda~X \highlight{{::} K}. t \quad\quad &type~abstraction\\
    \end{aligned}
\end{displaymath}
\end{mdframed}

\medskip

\begin{overprint}
\onslide<1>
The terms and values have not changed except for the extra kind annotation.
\onslide<2>
The same is true of the evaluation rules.
\end{overprint}

\end{frame}

\begin{frame}
  \begin{mdframed}[frametitle={Typing rules (changes from STLC)}]
  \infrule[T-Var]
  {\highlight[white]{x {:} T} \in \highlight[white]{\Gamma}}
  {\highlight[white]{\Gamma} \vdash x {:} T}

  \infrule[T-App]
  {\highlight[white]{\Gamma} \vdash t_1 {:} T_1 \rightarrow T_2 \andalso \highlight[white]{\Gamma} \vdash t_2 {:} T1}
  {\highlight[white]{\Gamma} \vdash t_1~t_2 {:} T_2}

  \infrule[T-Abs]
  {\highlight{\Gamma \vdash T_1 {::} *} \andalso \highlight[white]{\Gamma , x {:} T_1} \vdash t {:} T_2}
  {\highlight[white]{\Gamma} \vdash \left( \lambda~x {:} T_1 . t \right) {:} T_1 \rightarrow T_2}
  \end{mdframed}
  \medskip

  We need to make sure that our type annotations in lambda are types, rather
  than type operators.
\end{frame}

\begin{frame}
  \begin{mdframed}[frametitle={Typing rules (changes from System F)}]
  \infrule[T-TAbs]
  {\Gamma , X \highlight{{::} K_1} \vdash \highlight[white]{t_2} {:} \highlight[white]{T_2}}
  {\Gamma \vdash ( \lambda~X \higlight{{::} K_1} .
    \highlight[white]{t_2} ) {:} \forall~X \highlight{{::} K_1} . \highlight[white]{T_2} }

  \infrule[T-TApp]
  {\Gamma \vdash \highlight[white]{t_1~{:}~\forall~X~\highlight{{::}
        K_{11}}.~T_{12}} \andalso \highlight{\Gamma \vdash T_2 {::} K_{11}}}
  {\Gamma \vdash \highlight[white]{t_1 [{T_2}]} {:} \highlight[white]{[X \mapsto T_2 ] T_{12}} }
  \end{mdframed}
  \medskip

  We need to make sure that the kinds all match up properly with our universal types.
\end{frame}

\begin{frame}
  \begin{mdframed}[frametitle={Typing rules (the new one)}]
    \infrule[T-EQ]
    {\Gamma \vdash t {:} S \andalso S \equiv T \andalso \Gamma \vdash T {::} *}
    {\Gamma \vdash t {:} T}
  \end{mdframed}

  \medskip

  \begin{overprint}
    \onslide<1>
    If we allow type operators, we can write a type level ${id}$
    \[\text{Id} = \lambda~X~{::}~*~.~X\]
    \onslide<2>
    Given that, these all mean the same thing:
    \[\text{Nat} \rightarrow \text{Bool}\]
    \[\text{Id}~\text{Nat} \rightarrow \text{Bool}\]
    \[\text{Id}~\text{Nat} \rightarrow \text{Id}~\text{Bool}\]
    \[\text{Id}~(\text{Nat} \rightarrow \text{Bool})\]
    \onslide<3>
    We need a notion of type equivalence to deal with this.
    \onslide<4>
    $S \equiv T$ is our type equivalence relationship.
  \end{overprint}
\end{frame}

\begin{frame}
  \begin{mdframed}[frametitle={Typing equivalence rules (partial)}]
    \infrule[Q-REFL]
    {}
    {T \equiv T}
    \infrule[Q-SYM]
    {T \equiv S}
    {S \equiv T}
    \infrule[Q-TRANS]
    {S \equiv T \andalso T \equiv U}
    {S \equiv U}
    \infrule[Q-ARROW]
    {S_1 \equiv T_1 \andalso S_2 \equiv T_2}
    {S_1 \rightarrow S_2 \equiv T_1 \rightarrow T_2}
    \ldots 
  \end{mdframed}
  \medskip

  \begin{overprint}
    \onslide<1>
    Grah! It's a non-deterministic set of rules!
    \onslide<2>
    We should look at the kind system and see if it helps us out.
  \end{overprint}
\end{frame}

\begin{frame}
  \begin{mdframed}[frametitle={Kinding rules}]
    \begin{columns}
    \begin{column}{0.5\textwidth}

\begin{displaymath}
    \begin{aligned}
T \quad:=\quad& ~ \ldots \\
  & ~ X \quad\quad \\
  & ~ \lambda~X {::} K . T \\
  & ~ T ~ T \\
K \quad:=\quad& ~ &\\
   & ~ K \Rightarrow K 
    \end{aligned}
\end{displaymath}
    \end{column}
    \begin{column}{0.5\textwidth}
  \infrule[K-TVar]
  {X {::} K \in \Gamma}
  {\Gamma \vdash X {::} K}

  \infrule[K-App]
  {\Gamma \vdash T_1 {::} K_1 \rightarrow K_2 \andalso \Gamma \vdash T_2 {::} K_1}
  {\Gamma \vdash T_1~T_2 {::} K_2}

  \infrule[K-Abs]
  {\Gamma , X {::} K_1 \vdash T_2 {:} K_2}
  {\Gamma \vdash \left( \lambda~X {::} K_1 . T_2 \right) {::} K_1 \rightarrow K_2}
    \end{column}
    \end{columns}
  \end{mdframed}
  \medskip

  \begin{overprint}
    \onslide<1>
    This looks familiar...
    \onslide<2>
    It's simply typed lambda calculus, raised up one level!
    \onslide<3>
    That gives us high confidence in using our kind system to keep the types in check.
    \onslide<4>
    It also means we can drop the kind annotations and use our existing type
    inference algorithms to infer them!
    \onslide<5>
    On top of all that, it provides a hint about how to deal with the type equivalence problem.
    \onslide<6>
    STLC is strongly normalizing...
    \onslide<7>
    ... so we can use STLC evaluation rules to normalize our types before we
    compare them, giving us algorithmic type equivalence.
  \end{overprint}
\end{frame}

\begin{frame}
  \begin{mdframed}[frametitle={Kinding rules}]

    \infrule[K-Arrow]
    {\Gamma \vdash T_1 {::} * \andalso \Gamma \vdash T_2 {::} *}
    {\Gamma \vdash T_1 \rightarrow T_2 {::} *}
    
    \infrule[K-All]
    {\Gamma , X {::} K_1 \vdash T_2 {::} *}
    {\Gamma \vdash \forall X {::} K_1 . T_2 :: *}
    
  \end{mdframed}
  \medskip

  \begin{overprint}
    \onslide<1>
  We need a few more kinding rules to sanity check the other features in our
  type system.
    \onslide<2>
    Neither of these mess with the strong normalization property
    \onslide<3>
    Hurrah!
  \end{overprint}
\end{frame}

\begin{frame}
What does $\text{Pair}$ look like now?
\end{frame}

\begin{frame}
(Let us assume that we have set up a basic record system)
\end{frame}

\begin{frame}
  \[\lambda~A {::} * . \lambda~B {::} * . \{ \text{fst} : A, \text{snd} : B\} \]
\end{frame}
