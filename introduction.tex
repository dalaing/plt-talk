
\section{Introduction}

\begin{frame}
  Programming Language Theory (PLT) is handy to know about.
\end{frame}

\begin{frame}
  Useful for understanding how different languages work, comparing them, coming
  up with new languages.
\end{frame}

\begin{frame}
  {\it Very} useful to know if you want to read certain types of blog posts and papers.
\end{frame}

\begin{frame}
  You only need to know about a few things before you can read {\it many} more papers.
\end{frame}

\begin{frame}
  PLT is not as scary as it seems...
\end{frame}

\begin{frame}
  ... but it can seem quite scary at first.
\end{frame}

\begin{frame}[shrink=30]
  \begin{columns}[T]
    \begin{column}{0.5\textwidth}
  \begin{mdframed}[frametitle={Terms, values and types}]
\begin{displaymath}
    \begin{aligned}
t \quad:=\quad& &\\
  & ~ x \quad&variable\\
  & ~ \lambda~x {:} T . t \quad &abstraction\\
  & ~ t~t \quad&function~application\\
v \quad:=\quad& &\\
  & ~ \lambda~x {:} T . t \quad &abstraction\\
T \quad:=\quad& ~ &\\
  & ~ T \rightarrow T \quad &function~arrow\\
    \end{aligned}
\end{displaymath}
  \end{mdframed}
    \vfill
    \end{column}
    \begin{column}{0.5\textwidth}
  \begin{mdframed}[frametitle={Small-step semantics}]
  \infrule[E-App1]
  {t_1 \longrightarrow {t_1}^{\prime}}
  {t_1~t_2 \longrightarrow {t_1}^{\prime}~t_2}
  \infrule[E-App2]
  {t_2 \longrightarrow {t_2}^{\prime}}
  {v_1~t_2 \longrightarrow v_1~{t_2}^{\prime}}
  \infrule[E-AppAbs]
  {}
  {(\lambda~x {:} T . t_1) t_2 \longrightarrow \left[x \mapsto t_2 \right]t_1}
  \end{mdframed}
  \begin{mdframed}[frametitle={Typing rules}]
  \infrule[T-Var]
  {x {:} T \in \Gamma}
  {\Gamma \vdash x {:} T}

  \infrule[T-App]
  {\Gamma \vdash t_1 {:} T_1 \rightarrow T_2 \andalso \Gamma \vdash t_2 {:} T1}
  {\Gamma \vdash t_1~t_2 {:} T_2}

  \infrule[T-Abs]
  {\Gamma , x {:} T_1 \vdash T_2}
  {\Gamma \vdash \left( \lambda~x {:} T_1 . t \right) {:} T_1 \rightarrow T_2}
  \end{mdframed}
    \end{column}
  \end{columns}
\end{frame}

\begin{frame}[c]
  This talk is partly about getting you started with the notation, concepts and terminology.
\end{frame}

\begin{frame}[c]
  It will also cover how a lot of common language features are defined.
\end{frame}

\begin{frame}[c]
  It won't cover how to prove these various properties, although ``Types and
  Programming Languages'' and/or ``Software Foundations'' cover that very well
  if you are interested in that.
\end{frame}
