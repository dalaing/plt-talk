
\section{Universal types: System F}

\begin{frame}
  \begin{mdframed}[frametitle={Types}]
\begin{displaymath}
    \begin{aligned}
T \quad=\quad& ~ \ldots &\\
  & ~ \highlight{X} \quad\quad &type~variable\\
  & ~ T \rightarrow T \quad\quad &function~arrow\\
  & ~ \highlight{\forall~X . T} \quad\quad &univeral type\\
    \end{aligned}
\end{displaymath}
  \end{mdframed}
  We start by adding universal types into the mix
\end{frame}

\begin{frame}
  \begin{mdframed}[frametitle={Terms and values}]
\begin{displaymath}
    \begin{aligned}
t \quad=\quad& ~ \ldots &\\
  & ~ x \quad\quad &variable\\
  & ~ \lambda~x {:} T . t \quad\quad &abstraction\\
  & ~ t~t \quad\quad &function~application\\
  & ~ \highlight{\lambda~X . t} \quad\quad &type~abstraction\\
  & ~ \highlight{t~\left[T\right]} \quad\quad &type~application\\
v \quad=\quad& ~ \ldots &\\
  & ~ \lambda~x {:} T . t \quad\quad &abstraction\\
  & ~ \highlight{\lambda~X . t} \quad\quad &type~abstraction\\
    \end{aligned}
\end{displaymath}
  \end{mdframed}

  \medskip
  
  \begin{overprint}
    \onslide<1>
    We need the ability to abstract over a type, and to later supply a type.
    \onslide<2>
    We abstract over types to do things like $\text{id}$:
    \[\lambda~X~.~\lambda~x~{:}~X~.~x~{:}~\forall~A~.~A \rightarrow A\]
    \onslide<3>
    We supply concrete types later on to be able to use these terms:
    \[\left(\lambda~X~.~\lambda~x~{:}~X~.~x\right)~\left[ \text{Int} \right]~{:}~\text{Int} \rightarrow \text{Int}\]
    \onslide<4>
    If we didn't have these things to guide the way, we'd lose syntax-directed
    type checking / inference.
  \end{overprint}
  
\end{frame}

\begin{frame}
  \begin{mdframed}[frametitle={Small-step semantics (old)}]
  \infrule[E-App1]
  {t_1 \longrightarrow {t_1}^{\prime}}
  {t_1~t_2 \longrightarrow {t_1}^{\prime}~t_2}

  \infrule[E-App2]
  {t_2 \longrightarrow {t_2}^{\prime}}
  {v_1~t_2 \longrightarrow v_1~{t_2}^{\prime}}

  \infrule[E-AppAbs]
  {}
  {(\lambda~x {:} T . t_1) t_2 \longrightarrow \left[x \mapsto t_2 \right]t_1}
  \end{mdframed}
\end{frame}

\begin{frame}
  \begin{mdframed}[frametitle={Small-step semantics (new)}]

  \infrule[E-TApp]
  {t_1 \longrightarrow {t_1}^{\prime}}
  {t_1~\left[T_2\right] \longrightarrow {t_1}^{\prime}~\left[T_2\right]}
  
  \infrule[E-TAppTAbs]
  {}
  {(\lambda~X . t_{12}) \left[T_2\right] \longrightarrow \left[X \mapsto T_2 \right]t_{12}}

  \end{mdframed}

  \medskip
  
  \begin{overprint}
  \onslide<1>
  The new rules are straightforward.
  \onslide<2>
  $\text{E-TApp}$ says we evaluate the terms inside type applications.
  \onslide<3>
  $\text{E-TAppTAbs}$ says that when we find a type application is applied to a
  type abstraction, we carry out the a substitution wherever that type appears
  in the term $t_{i2}$.
  \onslide<4>
  In this case that will be in the substitutions will be in the type annotations on
  the lambda abstractions.
  \onslide<5>
  For example:
  \[
    \left(\lambda~X~.~\lambda~x~{:}~X~.~x\right)~\left[ \text{Int} \right]
    \longrightarrow
    \lambda~x~{:}~Int~.~x
  \]
  \end{overprint}
  
\end{frame}

\begin{frame}
  \begin{mdframed}[frametitle={Typing rules (old)}]
  \infrule[T-Var]
  {x {:} T \in \Gamma}
  {\Gamma \vdash x {:} T}

  \infrule[T-App]
  {\Gamma \vdash t_1 {:} T_1 \rightarrow T_2 \andalso \Gamma \vdash t_2 {:} T_1}
  {\Gamma \vdash t_1~t_2 {:} T_2}

  \infrule[T-Abs]
  {\Gamma , x {:} T_1 \vdash t {:} T_2}
  {\Gamma \vdash \left( \lambda~x {:} T_1 . t \right) {:} T_1 \rightarrow T_2}
  \end{mdframed}
\end{frame}

\begin{frame}
  \begin{mdframed}[frametitle={Typing rules (new)}]

  \begin{overprint}

  \onslide<1>
  \infrule[T-TAbs]
  {\Gamma , \highlight[white]{X} \vdash \highlight[white]{t_2} {:} \highlight[white]{T_2}}
  {\Gamma \vdash ( \lambda~\highlight[white]{X} . \highlight[white]{t_2} ) {:} \forall~\highlight[white]{X} . \highlight[white]{T_2} }
  \onslide<2>
  \infrule[T-TAbs]
  {\Gamma , \highlight[white]{X} \vdash \highlight[white]{t_2} {:} \highlight[white]{T_2}}
  {\Gamma \vdash ( \lambda~\highlight{X} . \highlight[white]{t_2} ) {:} \forall~\highlight{X} . \highlight[white]{T_2} }
  \onslide<3>
  \infrule[T-TAbs]
  {\Gamma , \highlight[white]{X} \vdash \highlight{t_2} {:} \highlight{T_2}}
  {\Gamma \vdash ( \lambda~\highlight[white]{X} . \highlight{t_2} ) {:} \forall~\highlight[white]{X} . \highlight{T_2} }
  \onslide<4>
  \infrule[T-TAbs]
  {\Gamma , \highlight{X} \vdash \highlight[white]{t_2} {:} \highlight[white]{T_2}}
  {\Gamma \vdash ( \lambda~\highlight[white]{X} . \highlight[white]{t_2} ) {:} \forall~\highlight[white]{X} . \highlight[white]{T_2} }
  \onslide<5->
  \infrule[T-TAbs]
  {\Gamma , \highlight[white]{X} \vdash \highlight[white]{t_2} {:} \highlight[white]{T_2}}
  {\Gamma \vdash ( \lambda~\highlight[white]{X} . \highlight[white]{t_2} ) {:} \forall~\highlight[white]{X} . \highlight[white]{T_2} }
  \end{overprint}

  \begin{overprint}
  \onslide<1-5>
  \infrule[T-TApp]
  {\Gamma \vdash \highlight[white]{t_1~{:}~\forall~X~.~T_{12}}}
  {\Gamma \vdash \highlight[white]{t_1 [{T_2}]} {:} \highlight[white]{[X \mapsto T_2 ] T_{12}} }

  \onslide<6>
  \infrule[T-TApp]
  {\Gamma \vdash \highlight{t_1~{:}~\forall~X~.~T_{12}}}
  {\Gamma \vdash \highlight[white]{t_1 [{T_2}]} {:} \highlight[white]{[X \mapsto T_2 ] T_{12}} }

  \onslide<7>
  \infrule[T-TApp]
  {\Gamma \vdash \highlight[white]{t_1~{:}~\forall~X~.~T_{12}}}
  {\Gamma \vdash \highlight{t_1 [{T_2}]} {:} \highlight[white]{[X \mapsto T_2 ] T_{12}} }

  \onslide<8>
  \infrule[T-TApp]
  {\Gamma \vdash \highlight[white]{t_1~{:}~\forall~X~.~T_{12}}}
  {\Gamma \vdash \highlight[white]{t_1 [{T_2}]} {:} \highlight{[X \mapsto T_2 ] T_{12}} }

  \end{overprint}

  \end{mdframed}

  \medskip

  \begin{overprint}
    \onslide<1>
    $\text{T-TAbs}$ gives type abstractions a universal type.
    \onslide<2>
    The type variable binding in the term becomes the binding in the universal type.
    \onslide<3>
    Other than that, the type doesn't change.
    \onslide<4> 
    The type variable is mentioned in the context, but we're not doing much with
    it (for now).
    \onslide<5> 
    $\text{T-TApp}$ consumes the universal types.
    \onslide<6> 
    We start with a term with a universal type.
    \onslide<7> 
    We apply a type to that term.
    \onslide<8> 
    And we get the type substitution that we were after.
  \end{overprint}
\end{frame}

\begin{frame}
  We should check on how $\text{const}$ and $\text{compose}$ are doing.
\end{frame}

\begin{frame}
  \[\lambda~X~.~\lambda~Y~.\]
  \[\lambda~x~{:}~X~.~\lambda~y~{:}~Y~.\]
  \[x\]
  \[{:}~\forall~A~.~\forall~B~.~A \rightarrow B \rightarrow A\]
\end{frame}

\begin{frame}
    \[\lambda~X~.~\lambda~Y~.~\lambda~Z~.\]
    \[\lambda~f~{:}~Y \rightarrow Z~.~\lambda~g~{:}~X \rightarrow Y~.~\lambda~x~{:}~X~.\]
    \[f~\left( g~x \right)\]
    \[{:}~\forall~A~.~\forall~B~.~\forall~C~.~\left( B \rightarrow C \right) \rightarrow \left( A \rightarrow B \right) \rightarrow \left( A \rightarrow C \right)\]
\end{frame}

\begin{frame}
  Type inference for System F is undecidable in general.
\end{frame}

\begin{frame}
  There is a whole new set of rabbit holes to dive into here.
\end{frame}

\begin{frame}
  The various options differ in whether they deal with higher rank types and
  impredicativity, and also in where you need to put annotations and how
  predictable the need for annotations is.
\end{frame}

\begin{frame}
  There seems to be two main general approaches.
\end{frame}

\begin{frame}
  1. Use fancier type schemes - see $\text{ML}^{\text{F}}$, $\text{HMF}$, $\text{HML}$.
\end{frame}

\begin{frame}
  2. Use bidirectional type checking / local type inference - make checking and
  inference mutually recursive so that you can propagate information from
  annotations to places where it might be needed.
\end{frame}

\begin{frame}
  System F is still pretty nice.
\end{frame}

\begin{frame}
  It would be nice if we didn't have to roll our own pairs and things as type
  system extensions...
\end{frame}
