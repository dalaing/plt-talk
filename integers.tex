
\section{Integers}

\begin{frame}[c]
  \begin{center}
    Let us look at a simple language, starting with the terms.
  \end{center}
\end{frame}

\begin{frame}
\begin{displaymath}
    \begin{aligned}
t \quad:=\quad& ~ &\\
  & ~ \left<\text{int}\right> \quad\quad &constant~integer\\
  & ~ t + t \quad\quad &addition\\
    \end{aligned}
  \end{displaymath}
\end{frame}

\begin{frame}[c]
    What do these terms look like?
\end{frame}

\begin{frame}
\begin{displaymath}
  3 
\end{displaymath}
\end{frame}

\begin{frame}
\begin{displaymath}
  (1 + 2) + (3 + 4)
\end{displaymath}
\end{frame}

\begin{frame}[c]
  When we e{\it valu}ate a term, we are turning it into a {\it value}.
\end{frame}

\begin{frame}[c]
  What are the values for the Integer language?
\end{frame}

\begin{frame}

  By rules:

  \infrule[V-Int]
          {}
          {\text{value}~\left<\text{int}\right>}
\end{frame}

\begin{frame}

  By definitions:
  
\begin{displaymath}
    \begin{aligned}
v \quad:=\quad& ~ &\\
  & ~ \left<\text{int}\right> \quad\quad &constant~integer\\
    \end{aligned}
  \end{displaymath}
\end{frame}

\begin{frame}[c]
  Evaluation proceeds in {\it steps}.
\end{frame}

\begin{frame}[c]
  The set of steps gives us the {\it small-step semantics} for the language.
\end{frame}

\begin{frame}[c]
  \begin{mdframed}[frametitle={Small-step semantics}]

  \infrule[E-Add1]
         {t_1 \longrightarrow {t_1}^{\prime}}
         {t_1 + t_2 \longrightarrow {t_1}^{\prime} + t_2}
  \infrule[E-Add2]
          {t_2 \longrightarrow {t_2}^{\prime}}
          {v_1 + t_2 \longrightarrow v_1 + {t_2}^{\prime}}
  \infrule[E-AddInt]
          {}
          {\left<\text{int}_1\right> + \left<\text{int}_2\right> \longrightarrow \left< \text{int}_1 + \text{int}_2 \right>}
  \end{mdframed}

  \medskip

  \begin{overprint}
  \onslide<1>
  The steps are specified as a binary relation $t_1 \longrightarrow t_2$.
  \onslide<2>
  The relation $t_1 \longrightarrow t_2$ indicates that the term $t_1$ can step to $t_2$.
  \onslide<3>
  $\text{E-AddInt}$ does the actual addition, $\text{E-Add1}$ and
  $\text{E-Add2}$ control the order in which the steps are applied to get to
  that point.
  \onslide<4>
  Any term that cannot take a step is known as a {\it normal form}.
  \onslide<5>
  Values cannot take a step by definition and so are always normal forms.
  \onslide<6>
  Iterating the small-step relation until you reach a value is called {\it evaluation}.
  \onslide<7>
  Iterating the small-step relation until you reach a normal form is called {\it normalization}.
  \onslide<8>
  Usually evaluation and normalization are / are hoped to be the same thing.
  \onslide<9>
  If a term is not a value but is a normal form, then it is {\it stuck}.
  \onslide<10>
  If all of the normal forms in our language are values, then the language is {\it normalizing}.
  \onslide<11>
  For a normalizing language: finite-sized terms will always evaluate in finite time.
  \onslide<12>
  The relationship between values and normal forms is a relationship between syntax and semantics.
  \end{overprint}
\end{frame}

\begin{frame}

  We can define evaluation in terms of the small-step relation:
  
  \infrule[Big-Value]
          {}
          {v \Rightarrow v}
  \infrule[Big-Step]
          {t \longrightarrow t^{\prime} \andalso t^{\prime} \Rightarrow v}
          {t \Rightarrow v}
\end{frame}

\begin{frame}
  Let us evaluate $(1 + 2) + (3 + 4)$.
\end{frame}

\begin{frame}
  \begin{mdframed}[frametitle={Previously...}]
  \infrule[E-Add1]
         {\highlight[white]{\strut t_1} \longrightarrow \highlight[white]{\strut {t_1}^{\prime}}}
         {\highlight[white]{\strut t_1} + \highlight[white]{\strut t_2} \longrightarrow \highlight[white]{\strut {t_1}^{\prime}} +
           \highlight[white]{\strut t_2}}
  \infrule[E-AddInt]
          {}
          {\highlight[white]{\strut \left<\text{int}_1\right>} + \highlight[white]{\strut
              \left<\text{int}_2\right>} \longrightarrow \highlight[white]{\strut \left< \text{int}_1 + \text{int}_2 \right>}}
  \end{mdframed}
  \vfill
  \begin{displaymath}
    \prftree[r]{E-Add1}
    {\prftree[r]{E-AddInt}
      {}
      {\highlight[white]{\strut 1} + \highlight[white]{\strut 2} \longrightarrow \highlight[white]{\strut 3}}
    }
    {\highlight[white]{\strut (1 + 2)} + \highlight[white]{\strut (3 + 4)}
      \longrightarrow \highlight[white]{\strut 3} + \highlight[white]{\strut (3 + 4)}}
  \end{displaymath}
\end{frame}

\begin{frame}
  \begin{mdframed}[frametitle={Previously...}]
  \infrule[E-Add1]
         {\highlight[white]{\strut t_1} \longrightarrow \highlight[white]{\strut {t_1}^{\prime}}}
         {\highlight[white]{\strut t_1} + \highlight[white]{\strut t_2} \longrightarrow \highlight[white]{\strut {t_1}^{\prime}} +
           \highlight[white]{\strut t_2}}
  \infrule[E-AddInt]
          {}
          {\highlight[white]{\strut \left<\text{int}_1\right>} + \highlight[white]{\strut
              \left<\text{int}_2\right>} \longrightarrow \highlight[white]{\strut \left< \text{int}_1 + \text{int}_2 \right>}}
  \end{mdframed}
  \vfill
  \begin{displaymath}
    \prftree[r]{E-Add1}
    {\prftree[r]{E-AddInt}
      {}
      {\highlight[white]{\strut 1} + \highlight[white]{\strut 2} \longrightarrow \highlight[white]{\strut 3}}
    }
    {\highlight[violet]{\strut (1 + 2)} + \highlight[blue]{\strut (3 + 4)}
      \longrightarrow \highlight[purple]{\strut 3} + \highlight[blue]{\strut (3 + 4)}}
  \end{displaymath}
\end{frame}

\begin{frame}
  The complete evaluation takes three steps. \\

  First:
  \begin{displaymath}
    \prftree[r]{E-Add1}
    {\prftree[r]{E-AddInt}
      {}
      {1 + 2 \longrightarrow 3}
    }
    {(1 + 2) + (3 + 4) \longrightarrow 3 + (3 + 4)}
  \end{displaymath}

  Then:
  \begin{displaymath}
    \prftree[r]{E-Add2}
    {\prftree[r]{E-AddInt}
      {}
      {3 + 4 \longrightarrow 7}
    }
    {3 + (3 + 4) \longrightarrow 3 + 7}
  \end{displaymath}

  Finally:
  \begin{displaymath}
    \prftree[r]{E-AddInt}
    {}
    {3 + 7 \longrightarrow 10}
  \end{displaymath}
\end{frame}
