
\section{Booleans}

\begin{frame}
  \begin{mdframed}[frametitle={Terms and values}]
\begin{displaymath}
    \begin{aligned}
t \quad:=\quad& ~ &\\
  & ~ \text{false} \quad\quad &constant~false\\
  & ~ \text{true} \quad\quad &constant~true\\
  & ~ t ~ \text{or} ~ t \quad\quad &disjunction\\
v \quad:=\quad& ~ &\\
  & ~ \text{false} \quad\quad &false~value \\
  & ~ \text{true} \quad\quad &true~value \\
    \end{aligned}
  \end{displaymath}
  \end{mdframed}
\end{frame}

\begin{frame}
  \begin{mdframed}[frametitle={Small-step semantics}]
  \infrule[E-Or1]
         {t_1 \longrightarrow {t_1}^{\prime}}
         {t_1~\text{or}~t_2 \longrightarrow {t_1}^{\prime}~\text{or}~t_2}
  \infrule[E-Or2]
          {t_2 \longrightarrow {t_2}^{\prime}}
          {v_1~\text{or}~t_2 \longrightarrow v_1~\text{or}~{t_2}^{\prime}}
  \infrule[E-OrFalseFalse]
          {}
          {\text{false}~\text{or}~\text{false} \longrightarrow \text{false}}
  \infrule[E-OrFalseTrue]
          {}
          {\text{false}~\text{or}~\text{true} \longrightarrow \text{true}}
  \infrule[E-OrTrueFalse]
          {}
          {\text{true}~\text{or}~\text{false} \longrightarrow \text{true}}
  \infrule[E-OrTrueTrue]
          {}
          {\text{true}~\text{or}~\text{true} \longrightarrow \text{true}}
  \end{mdframed}
\end{frame}

\begin{frame}[c]
  We should probably short-circuit the evaluation...
\end{frame}

\begin{frame}[c]
  \begin{mdframed}[frametitle={Small-step semantics}]
  \infrule[E-Or1]
         {t_1 \longrightarrow {t_1}^{\prime}}
         {t_1~\text{or}~t_2 \longrightarrow {t_1}^{\prime}~\text{or}~t_2}
  \infrule[E-OrFalse]
          {}
          {\text{false}~\text{or}~t_2 \longrightarrow t_2}
  \infrule[E-OrTrue]
          {}
          {\text{true}~\text{or}~t_2 \longrightarrow \text{true}}
  \end{mdframed}

  \medskip

  \begin{overprint}
  \onslide<+>
  \onslide<+>
  In the languages we've seen so far, at most one rule applies to each term.
  \onslide<+>
  If the rules for the steps overlap for a particular term, but the result of the step is the same for all of the overlapping rules, then all is well.
  \onslide<+>
  This means that we don't have to worry about the order in which the rules are applied.
  \onslide<+>
  In other cases where rules overlap, they can be harder to deal with.
  \onslide<+>
  If we had the short-circuiting rules and the non-short circuiting rules in use
  at the same time, we would have some troubles.
  \onslide<+>
  If we non-deterministically choose a rule to apply we will end up stepping to
  different terms, but evaluation will end up at the same value.
  \onslide<+>
  Things could be worse.
  \end{overprint}
\end{frame}

\begin{frame}[c,shrink=20]
  \begin{mdframed}[frametitle={A non-deterministic set of rules}]
    \begin{columns}
      \begin{column}{0.5\textwidth}
  \infrule[Eq-Refl]{}{t = t}
  \infrule[Eq-Sym]{t_2 = t_1}{t_1 = t_2}
  \infrule[Eq-Trans]{t_1 = t_2 \andalso t_2 = t_3}{t_1 = t_3}
      \end{column}
      \begin{column}{0.5\textwidth}
  \infrule[Eq-Or]{t_1 = t_2 \andalso t_3 = t_4}{t_1~\text{or}~t_3 = t_2~\text{or}~t_4}
  \infrule[Eq-OrFalse]{}{\text{false}~\text{or}~t = t}
  \infrule[Eq-OrTrue]{}{\text{true}~\text{or}~t = \text{true}}
      \end{column}
    \end{columns}
  \end{mdframed}

  \medskip

  \begin{overprint}
  \onslide<+>
  Some rules have multiple valid choices for some terms that can create loops.
  \onslide<+>
  These are hard to implement, since you could get stuck applying

  \infrule[Eq-Sym]{t_2 = t_1}{t_1 = t_2}

  over and over.
  \onslide<+>
  Normally the non-deterministic rules are there to make propositions or proofs easier to work with.
  \onslide<+>
  Most of the time there will also be a corresponding set of deterministic rules - often referred to as {\it algorithmic} - which will aid the implementors.
  \onslide<+>
  This is usually follow by a proof of equivalence between the non-deterministic and algorithmic rule sets.
  \end{overprint}
\end{frame}

