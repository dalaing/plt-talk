
\section{Natural numbers}

\begin{frame}[c]
  A natural number is either zero or the successor of a natural number
\end{frame}

\begin{frame}[c]
  This is more or less working in a unary number system
\end{frame}

\begin{frame}[c]
  \begin{displaymath}
    3 \equiv succ~succ~succ~O
  \end{displaymath}
\end{frame}

\begin{frame}[c]
  We can have eager natural numbers or lazy natural numbers
\end{frame}

\begin{frame}
  \begin{mdframed}[frametitle={Terms and values (eager)}]
\begin{displaymath}
    \begin{aligned}
t \quad:=\quad& ~ &\\
  & ~ \text{O} \quad\quad &constant~zero \\
  & ~ \text{succ}~t \quad\quad &successor \\
  & ~ \text{pred}~t \quad\quad &predecessor \\
v \quad:=\quad& ~ &\\
  & ~ \text{O} \quad\quad &zero~value \\
  & ~ \text{succ}~\highlight{v} \quad\quad &successor~value \\
    \end{aligned}
  \end{displaymath}
  \end{mdframed}
\end{frame}

\begin{frame}
  \begin{mdframed}[frametitle={Terms and values (lazy)}]
\begin{displaymath}
    \begin{aligned}
t \quad:=\quad& ~ &\\
  & ~ \text{O} \quad\quad &constant~zero \\
  & ~ \text{succ}~t \quad\quad &successor \\
  & ~ \text{pred}~t \quad\quad &predecessor \\
v \quad:=\quad& ~ &\\
  & ~ \text{O} \quad\quad &zero~value \\
  & ~ \text{succ}~\highlight{t} \quad\quad &successor~value \\
    \end{aligned}
  \end{displaymath}
  \end{mdframed}
\end{frame}

\begin{frame}
  \begin{mdframed}[frametitle={Small-step semantics}]

  \infrule[$\text{E-Succ}^{\text{*}}$]
         {t_1 \longrightarrow {t_1}^{\prime}}
         {\text{succ}~t_1 \longrightarrow \text{succ}~{t_1}^{\prime}}
  \infrule[E-Pred]
          {t_1 \longrightarrow {t_1}^{\prime}}
          {\text{pred}~t_1 \longrightarrow \text{pred}~{t_1}^{\prime}}
  \infrule[E-PredZero]
          {}
          {\text{pred}~\text{O} \longrightarrow \text{O}}
  \infrule[E-PredSucc]
          {}
          {\text{pred} ~ \left( \text{succ}~v \right) \longrightarrow v}

  * Only for eager evaluation
  \end{mdframed}
\end{frame}

\begin{frame}[c]
  These are the same natural numbers under eager evaluation:
  
  \[\text{succ}~O\]

  and under lazy evaluation:

  \[\text{succ} \left( \text{pred} \left( \text{succ}~O \right) \right)\]
\end{frame}

\begin{frame}[c]
  Under eager evaluation, we don't want our values to have anything in them that
  needs to take a step.
\end{frame}

\begin{frame}[c]
  Under lazy evaluation, we don't want to take any steps that we don't need to.
\end{frame}

\begin{frame}[c]
  This is fine:

  \[\text{succ} \left( \text{pred} \left( \text{succ}~O \right) \right)\]

  because a natural number is either zero or the successor of a natural number.
\end{frame}

\begin{frame}[c]
  If we use $\text{pred}$  on this:

  \[\text{succ} \left( \text{pred} \left( \text{succ}~O \right) \right)\]

  then the outer $\text{succ}$ will be removed and evaluation will continue
  until we hit a $\text{O}$ or end up with another $\text{succ}$ on the outside.
\end{frame}