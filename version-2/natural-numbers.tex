
\section{Natural numbers}

\begin{frame}[c]
  A natural number is either zero or the successor of a natural number.
\end{frame}

\begin{frame}[c]
  This is a unary number system.
\end{frame}

\begin{frame}
  \begin{mdframed}[frametitle={Terms and values (eager)}]
\begin{displaymath}
    \begin{aligned}
t \quad:=\quad& ~ &\\
  & ~ \text{O} \quad\quad &constant~zero \\
  & ~ \text{succ}~t \quad\quad &successor \\
  & ~ \text{pred}~t \quad\quad &predecessor \\
v \quad:=\quad& ~ &\\
  & ~ \text{O} \quad\quad &zero~value \\
  & ~ \text{succ}~\highlight{v} \quad\quad &successor~value \\
    \end{aligned}
  \end{displaymath}
  \end{mdframed}
\end{frame}

\begin{frame}
  \begin{mdframed}[frametitle={Terms and values (lazy)}]
\begin{displaymath}
    \begin{aligned}
t \quad:=\quad& ~ &\\
  & ~ \text{O} \quad\quad &constant~zero \\
  & ~ \text{succ}~t \quad\quad &successor \\
  & ~ \text{pred}~t \quad\quad &predecessor \\
v \quad:=\quad& ~ &\\
  & ~ \text{O} \quad\quad &zero~value \\
  & ~ \text{succ}~\highlight{t} \quad\quad &successor~value \\
    \end{aligned}
  \end{displaymath}
  \end{mdframed}
\end{frame}

\begin{frame}[c]
  \begin{displaymath}
  \text{succ}~O
  \end{displaymath}
\end{frame}

\begin{frame}[c]
  \begin{displaymath}
  \text{succ} \left( \text{pred} \left( \text{succ}~O \right) \right)
  \end{displaymath}
\end{frame}

\begin{frame}
  \begin{mdframed}[frametitle={Small-step semantics (eager)}]

  \infrule[E-Succ]
         {\highlight{t_1 \longrightarrow {t_1}^{\prime}}}
        {\highlight{\text{succ}~t_1 \longrightarrow \text{succ}~{t_1}^{\prime}}}
  \infrule[E-Pred]
          {t_1 \longrightarrow {t_1}^{\prime}}
          {\text{pred}~t_1 \longrightarrow \text{pred}~{t_1}^{\prime}}
  \infrule[E-PredZero]
          {}
          {\text{pred}~\text{O} \longrightarrow \text{O}}
  \infrule[E-PredSucc]
          {}
          {\text{pred} ~ \left( \text{succ}~\highlight{v} \right) \longrightarrow \highlight{v}}
  \end{mdframed}
\end{frame}

\begin{frame}
  \begin{mdframed}[frametitle={Small-step semantics (lazy)}]

  \invisible{\infrule[E-Succ]
        {t_1 \longrightarrow {t_1}^{\prime}}
        {\text{succ}~t_1 \longrightarrow \text{succ}~{t_1}^{\prime}}}
  \infrule[E-Pred]
          {t_1 \longrightarrow {t_1}^{\prime}}
          {\text{pred}~t_1 \longrightarrow \text{pred}~{t_1}^{\prime}}
  \infrule[E-PredZero]
          {}
          {\text{pred}~\text{O} \longrightarrow \text{O}}
  \infrule[E-PredSucc]
          {}
          {\text{pred} ~ \left( \text{succ}~\highlight{t} \right) \longrightarrow \highlight{t}}
  \end{mdframed}
\end{frame}

\begin{frame}[c]
  These are both versions of the number 1.

  \vfill

  With eager evaluation:
  \begin{displaymath}
  \text{succ}~O
  \end{displaymath}

  With lazy evaluation:
  \begin{displaymath}
  \text{succ} \left( \text{pred} \left( \text{succ}~O \right) \right)
  \end{displaymath}
\end{frame}

\begin{frame}[c]
  Under eager evaluation, we don't want our values to have anything in them that needs to take a step.
\end{frame}

\begin{frame}[c]
  Under lazy evaluation, we don't want to take any steps that we don't need to.
\end{frame}

\begin{frame}[c]
  This is fine:

  \begin{displaymath}
  \text{succ} \left( \text{pred} \left( \text{succ}~O \right) \right)
  \end{displaymath}

  because a natural number is either zero or the successor of a natural number.
\end{frame}

\begin{frame}[c]
  If we use $\text{pred}$  on this:

  \begin{displaymath}
  \text{succ} \left( \text{pred} \left( \text{succ}~O \right) \right)
  \end{displaymath}

  then the outer $\text{succ}$ will be removed and evaluation will continue
  until we hit a $\text{O}$ or end up with another $\text{succ}$ on the outside.
\end{frame}

\begin{frame}[c]
  If we don't use $\text{pred}$  on this:

  \begin{displaymath}
  \text{succ} \left( \text{pred} \left( \text{succ}~O \right) \right)
  \end{displaymath}

  then no one cared about it anyhow, so nothing of value is lost.
\end{frame}

\begin{frame}[fragile]
  \begin{minted}{haskell}
data Term =
    TmZero
  | TmSucc Term
  | TmPred Term
  deriving (Eq, Ord, Show)
  \end{minted}
\end{frame}

\begin{frame}[fragile]
  \begin{columns}
    \begin{column}{0.5\textwidth}
      \begin{minted}{haskell}
vZero :: Rule Term ()
vZero _ TmZero =
  Just ()
vZero _ _ =
  Nothing
      \end{minted}
    \end{column}
    \begin{column}{0.5\textwidth}
  \infrule[V-Zero]
         {}
        {\text{value} ~ \text{O}}
    \end{column}
  \end{columns}
\end{frame}

\begin{frame}[fragile]
  \begin{columns}
    \begin{column}{0.5\textwidth}
      \begin{minted}{haskell}
vSuccEager :: Rule Term ()
vSuccEager value (TmSucc tm) = do
  _ <- value tm
  pure ()
vSuccEager _ _ =
  Nothing
      \end{minted}
    \end{column}
    \begin{column}{0.5\textwidth}
  \infrule[V-Succ]
         {}
        {\text{value} ~ \left(\text{succ}~\text{v}\right)}
    \end{column}
  \end{columns}
\end{frame}

\begin{frame}[fragile]
  \begin{minted}{haskell}
valueEagerR :: RuleSet Term ()
valueEagerR =
  mkRuleSet [vZero, vSuccEager]
  \end{minted}
\end{frame}

\begin{frame}[fragile]
  \begin{columns}
    \begin{column}{0.5\textwidth}
      \begin{minted}{haskell}
vSuccLazy :: Rule Term ()
vSuccLazy _ (TmSucc tm) =
  pure ()
vSuccLazy _ _ =
  Nothing
      \end{minted}
    \end{column}
    \begin{column}{0.5\textwidth}
  \infrule[V-Succ]
         {}
        {\text{value} ~ \left(\text{succ}~\text{t}\right)}
    \end{column}
  \end{columns}
\end{frame}

\begin{frame}[fragile]
  \begin{minted}{haskell}
valueLazyR :: RuleSet Term ()
valueLazyR =
  mkRuleSet [vZero, vSuccLazy]
  \end{minted}
\end{frame}

\begin{frame}[fragile]
  \begin{columns}
    \begin{column}{0.5\textwidth}
      \begin{minted}{haskell}
eSucc :: Rule Term Term
eSucc step (TmSucc tm) = do
  tm' <- step tm
  pure $ TmSucc tm'
eSucc _ _ =
  Nothing
      \end{minted}
    \end{column}
    \begin{column}{0.5\textwidth}
  \infrule[E-Succ]
         {t_1 \longrightarrow {t_1}^{\prime}}
        {\text{succ}~t_1 \longrightarrow \text{succ}~{t_1}^{\prime}}
    \end{column}
  \end{columns}
\end{frame}

\begin{frame}[fragile]
  \begin{columns}
    \begin{column}{0.5\textwidth}
      \begin{minted}{haskell}
ePred :: Rule Term Term
ePred step (TmPred tm) = do
  tm' <- step tm
  pure $ TmPred tm'
ePred _ _ =
  Nothing
      \end{minted}
    \end{column}
    \begin{column}{0.5\textwidth}
  \infrule[E-Pred]
          {t_1 \longrightarrow {t_1}^{\prime}}
          {\text{pred}~t_1 \longrightarrow \text{pred}~{t_1}^{\prime}}
    \end{column}
  \end{columns}
\end{frame}

\begin{frame}[fragile]
  \begin{columns}
    \begin{column}{0.5\textwidth}
      \begin{minted}{haskell}
ePredZero :: Rule Term Term
ePredZero _ (TmPred TmZero) =
  Just TmZero
ePredZero _ _ =
  Nothing
      \end{minted}
    \end{column}
    \begin{column}{0.5\textwidth}
  \infrule[E-PredZero]
          {}
          {\text{pred}~\text{O} \longrightarrow \text{O}}
    \end{column}
  \end{columns}
\end{frame}

\begin{frame}[fragile]
  \begin{columns}
    \begin{column}{0.5\textwidth}
      \begin{minted}{haskell}
ePredSuccEager :: RuleSet Term () -> Rule Term Term
ePredSuccEager value _ (TmPred (TmSucc tm)) = do
  _ <- value tm
  pure tm
ePredSuccEager _ _ _ =
  Nothing
      \end{minted}
    \end{column}
    \begin{column}{0.5\textwidth}
  \infrule[E-PredSucc]
          {}
          {\text{pred} ~ \left( \text{succ}~v \right) \longrightarrow v}
    \end{column}
  \end{columns}
\end{frame}

\begin{frame}[fragile]
  \begin{minted}{haskell}
stepEagerR :: RuleSet Term Term
stepEagerR =
  mkRuleSet [eSucc, ePred, ePredZero, ePredSuccEager valueEagerR]
  \end{minted}
\end{frame}

\begin{frame}[fragile]
  \begin{columns}
    \begin{column}{0.5\textwidth}
      \begin{minted}{haskell}
ePredSuccLazy :: Rule Term Term
ePredSuccLazy _ (TmPred (TmSucc tm)) =
  Just tm
ePredSuccLazy _ _ =
  Nothing
      \end{minted}
    \end{column}
    \begin{column}{0.5\textwidth}
  \infrule[E-PredSucc]
          {}
          {\text{pred} ~ \left( \text{succ}~t \right) \longrightarrow t}
    \end{column}
  \end{columns}
\end{frame}

\begin{frame}[fragile]
  \begin{minted}{haskell}
stepLazyR :: RuleSet Term Term
stepLazyR =
  mkRuleSet [ePred, ePredZero, ePredSuccLazy]
  \end{minted}
\end{frame}

\begin{frame}[fragile]
  \onslide<+->
  \begin{minted}{haskell}
> let tm = TmSucc (TmPred (TmSucc TmZero))
  \end{minted}
  \onslide<+->
  \begin{minted}{haskell}
> stepEagerR tm
  \end{minted}
  \onslide<+->
  \begin{minted}{haskell}
Just (TmSucc TmZero)
  \end{minted}
  \onslide<+->
  \begin{minted}{haskell}
> stepEagerR >=> stepEagerR $ tm
  \end{minted}
  \onslide<+->
  \begin{minted}{haskell}
Nothing
  \end{minted}
\end{frame}

\begin{frame}[fragile]
  \onslide<+->
  \begin{minted}{haskell}
> let tm = TmSucc (TmPred (TmSucc TmZero))
  \end{minted}
  \onslide<+->
  \begin{minted}{haskell}
> stepLazyR tm
  \end{minted}
  \onslide<+->
  \begin{minted}{haskell}
Nothing
  \end{minted}
\end{frame}

\begin{frame}[fragile]
  \onslide<+->
  \begin{minted}{haskell}
> let tm = TmSucc (TmPred (TmSucc TmZero))
  \end{minted}
  \onslide<+->
  \begin{minted}{haskell}
> stepEagerR $ TmPred tm
  \end{minted}
  \onslide<+->
  \begin{minted}{haskell}
Just (TmPred (TmSucc TmZero))
  \end{minted}
  \onslide<+->
  \begin{minted}{haskell}
> stepEagerR >=> stepEagerR $ TmPred tm
  \end{minted}
  \onslide<+->
  \begin{minted}{haskell}
Just TmZero
  \end{minted}
  \onslide<+->
  \begin{minted}{haskell}
> stepEagerR >=> stepEagerR >=> stepEagerR $ TmPred tm
  \end{minted}
  \onslide<+->
  \begin{minted}{haskell}
Nothing
  \end{minted}
\end{frame}

\begin{frame}[fragile]
  \onslide<+->
  \begin{minted}{haskell}
> let tm = TmSucc (TmPred (TmSucc TmZero))
  \end{minted}
  \onslide<+->
  \begin{minted}{haskell}
> stepLazyR $ TmPred tm
  \end{minted}
  \onslide<+->
  \begin{minted}{haskell}
Just (TmPred (TmSucc TmZero))
  \end{minted}
  \onslide<+->
  \begin{minted}{haskell}
> stepLazyR >=> stepLazyR $ TmPred tm
  \end{minted}
  \onslide<+->
  \begin{minted}{haskell}
Just TmZero
  \end{minted}
  \onslide<+->
  \begin{minted}{haskell}
> stepLazyR >=> stepLazyR >=> stepLazyR $ TmPred tm
  \end{minted}
  \onslide<+->
  \begin{minted}{haskell}
Nothing
  \end{minted}
\end{frame}
