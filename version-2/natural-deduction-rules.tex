
\section{Natural deduction rules}

\begin{frame}[c]
  We use natural deduction style rules for most of this.
\end{frame}

\begin{frame}
  \infrule[Rule-Name]
          {Assumption_1 \andalso Assumption_2 \andalso \ldots}
          {Conclusion}
\end{frame}

\begin{frame}[c]
  \begin{mdframed}[frametitle={Rules for evenness}]

  \infrule[Even-Zero]
          {}
          {\text{even}~0}
  \infrule[Even-Add]
          {\text{even}~x}
          {\text{even}~\left(x + 2\right)}
  \end{mdframed}

  \medskip

  \begin{overprint}
  \onslide<+>
  Some rules - like $\text{Even-Zero}$ - have no assumptions, and are known as {\it axioms}.
  \onslide<+>
  Some rules - like $\text{Even-Add}$ - are recursive.
  \onslide<+>
  The relations are determined by the union of all of the rules.
  \onslide<+>
  Usually only one rule will apply at one time, and so the order we organize these rules in doesn't matter.
  \onslide<+>
  We are also dealing with an "open world".
  \onslide<+>
  Rules will often get added to a system without having to go back and alter the other rules.
  \end{overprint}
\end{frame}

\begin{frame}[c]
  \begin{mdframed}[frametitle={Rules for evenness, now with bonus rules}]

  \infrule[Even-Zero]
          {}
          {\text{even}~0}
  \infrule[Even-Add]
          {\text{even}~x}
          {\text{even}~\left(x + 2\right)}
  \infrule[Even-Odd]
          {\text{odd}~x}
          {\text{even}~\left(x + 1\right)}
  \infrule[Odd-Even]
          {\text{even}~x}
          {\text{odd}~\left(x + 1\right)}
  \end{mdframed}

  \medskip

  \begin{overprint}
  \onslide<+>
  Let us add some more rules.
  \onslide<+>
  Now we could apply these rules in a few different orders to determine $\text{even 4}$.
  \onslide<+>
  We want the rules to be {\it deterministic} - the answers through all of the paths agree, and the paths are finite.
  \end{overprint}
\end{frame}

\begin{frame}[c]
  \begin{mdframed}[frametitle={Equivalence relations}]

  \infrule[Reflexivity]
          {}
          {P = P}
  \infrule[Symmetry]
          {Q = P}
          {P = Q}
  \infrule[Transitivity]
          {P = Q \andalso Q = R}
          {P = R}
  \end{mdframed}

  \medskip

  \begin{overprint}
  \onslide<+>
  Sometimes we have rules like $\text{Symmetry}$, which could be applied over and over and spin forever.
  \onslide<+>
  Those rules usually exist in that form to explain why a system has some desired properties, or to assist with proofs.
  \onslide<+>
  There will often be a second equivalent set of rules applied that are deterministic and terminating - known as {\i algorithmic} to assist with implementations.
  \onslide<+>
  This is usually proved by a proof of logical equivalence of the non-deterministic and algorithmic rules sets.
  \end{overprint}
\end{frame}

