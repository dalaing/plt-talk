
\section{Integers}

\begin{frame}[c]
  \begin{center}
    Let us look at a simple language, starting with the terms.
  \end{center}
\end{frame}

\begin{frame}
\begin{mdframed}[frametitle={Terms}]
\begin{displaymath}
    \begin{aligned}
t \quad:=\quad& ~ &\\
  & ~ \left<\text{int}\right> \quad\quad &constant~integer\\
  & ~ t + t \quad\quad &addition\\
    \end{aligned}
  \end{displaymath}
\end{mdframed}

  \medskip

\begin{overprint}
\onslide<+>
These are the pieces of the language and the ways those pieces can be combined.
\onslide<+>
This is the abstract syntax of our language.
\end{overprint}
\end{frame}

\begin{frame}
\begin{displaymath}
  3 
\end{displaymath}
\end{frame}

\begin{frame}
\begin{displaymath}
  (1 + 2) + (3 + 4)
\end{displaymath}
\end{frame}

\begin{frame}[fragile]
  \begin{columns}
    \begin{column}{0.5\textwidth}
  \begin{minted}{haskell}
data Term =
    TmInt Int
  | TmAdd Term Term
  deriving (Eq, Ord, Show)
  \end{minted}
    \end{column}
    \begin{column}{0.5\textwidth}
    \begin{displaymath}
    \begin{aligned}
t \quad:=\quad& ~ &\\
  & ~ \left<\text{int}\right> \quad\quad &constant~integer\\
  & ~ t + t \quad\quad &addition\\
    \end{aligned}
    \end{displaymath}
    \end{column}
  \end{columns}
\end{frame}

\begin{frame}[fragile]
  \begin{columns}
    \begin{column}{0.5\textwidth}
  \begin{minted}{haskell}
TmInt 3
  \end{minted}
    \end{column}
    \begin{column}{0.5\textwidth}
    \begin{displaymath}
3
    \end{displaymath}
    \end{column}
  \end{columns}
\end{frame}

\begin{frame}[fragile]
  \begin{columns}
    \begin{column}{0.5\textwidth}
  \begin{minted}{haskell}
TmAdd (TmAdd (TmInt 1) (TmInt 2)) (TmInt 3)
  \end{minted}
    \end{column}
    \begin{column}{0.5\textwidth}
    \begin{displaymath}
(1 + 2) + 3
    \end{displaymath}
    \end{column}
  \end{columns}
\end{frame}

\begin{frame}[c]
  What are the values for the Integer language?
\end{frame}

\begin{frame}
\begin{mdframed}[frametitle={By definitions:}]
\begin{displaymath}
    \begin{aligned}
v \quad:=\quad& ~ &\\
  & ~ \left<\text{int}\right> \quad\quad &constant~integer\\
    \end{aligned}
  \end{displaymath}
\end{mdframed}

\medskip

  We define the values in terms of the syntax of the language.
\end{frame}

\begin{frame}
\begin{mdframed}[frametitle={By rules:}]
  \infrule[V-Int]
          {}
          {\text{value}~\left<\text{int}\right>}
\end{mdframed}
\medskip
Having the rules can help with the implementation.
\end{frame}

\begin{frame}[c]
  When we e{\it valu}ate a term, we are turning it into a {\it value}.
\end{frame}

\begin{frame}[c]
    Values are specified as part of the {\it syntax} of a languge.
\end{frame}

\begin{frame}[c]
  Evaluation proceeds in {\it steps}.
\end{frame}

\begin{frame}[c]
  The set of steps gives us the {\it small-step semantics} for the language.
\end{frame}

\begin{frame}[c]
  The steps are specified as a binary relation $t_1 \longrightarrow t_2$.
\end{frame}

\begin{frame}[c]
  The relation $t_1 \longrightarrow t_2$ indicates that the term $t_1$ can step to $t_2$.
\end{frame}

\begin{frame}[c]
  \begin{mdframed}[frametitle={Small-step semantics (partial)}]
  \infrule[E-AddInt]
          {}
          {\left<\text{int}_1\right> + \left<\text{int}_2\right> \longrightarrow \left< \text{int}_1 + \text{int}_2 \right>}
  \end{mdframed}

  \medskip

  $\text{E-AddInt}$ does the actual addition.
\end{frame}

\begin{frame}[c]
  \begin{mdframed}[frametitle={Small-step semantics (partial)}]

  \infrule[E-Add1]
         {t_1 \longrightarrow {t_1}^{\prime}}
         {t_1 + t_2 \longrightarrow {t_1}^{\prime} + t_2}
  \infrule[E-Add2]
          {\text{value}~t_1 \andalso t_2 \longrightarrow {t_2}^{\prime}}
          {t_1 + t_2 \longrightarrow t_1 + {t_2}^{\prime}}
  \end{mdframed}

  \medskip

  $\text{E-Add1}$ and $\text{E-Add2}$ control the order in which the steps are
  applied to get to the point where $\text{E-AddInt}$ applies.
\end{frame}

\begin{frame}[c]
  \begin{mdframed}[frametitle={Small-step semantics (partial)}]

  \infrule[E-Add1]
         {t_1 \longrightarrow {t_1}^{\prime}}
         {t_1 + t_2 \longrightarrow {t_1}^{\prime} + t_2}
  \infrule[E-Add2]
          {t_2 \longrightarrow {t_2}^{\prime}}
          {v_1 + t_2 \longrightarrow v_1 + {t_2}^{\prime}}
  \end{mdframed}

  \medskip

  We can use conventions of notation to simplify references to terms that should be values.
\end{frame}

\begin{frame}[c]
  \begin{mdframed}[frametitle={Small-step semantics}]

  \infrule[E-Add1]
         {t_1 \longrightarrow {t_1}^{\prime}}
         {t_1 + t_2 \longrightarrow {t_1}^{\prime} + t_2}
  \infrule[E-Add2]
          {t_2 \longrightarrow {t_2}^{\prime}}
          {v_1 + t_2 \longrightarrow v_1 + {t_2}^{\prime}}
  \infrule[E-AddInt]
          {}
          {\left<\text{int}_1\right> + \left<\text{int}_2\right> \longrightarrow \left< \text{int}_1 + \text{int}_2 \right>}
  \end{mdframed}
\end{frame}

\begin{frame}[c]
  Any term that cannot take a step is known as a {\it normal form}.
\end{frame}

\begin{frame}[c]
  Values cannot take a step by definition and so are always normal forms.
\end{frame}

\begin{frame}[c]
  Iterating the small-step relation until you reach a value is called {\it evaluation}.
\end{frame}

\begin{frame}[c]
  Iterating the small-step relation until you reach a normal form is called {\it normalization}.
\end{frame}

\begin{frame}[c]
  Usually evaluation and normalization are / are hoped to be the same thing.
\end{frame}

\begin{frame}[c]
  If a term is not a value but is a normal form, then it is {\it stuck}.
\end{frame}

\begin{frame}[c]
  A language can be {\it normalizing}: there is an evaluation order that means that finite-sized terms will always evaluate in finite time.
\end{frame}

\begin{frame}[c]
  A language can be {\it strongly normalizing}: for any evaluation order, finite-sized terms will always evaluate in finite time.
\end{frame}

\begin{frame}[c]
  The relationship between values and normal forms is a relationship between syntax and semantics.
\end{frame}

\begin{frame}

  Aside: We can define evaluation in terms of a big-step relation:
  
  \infrule[Big-Value]
          {}
          {v \Rightarrow v}
  \infrule[Big-Step]
          {t \longrightarrow t^{\prime} \andalso t^{\prime} \Rightarrow v}
          {t \Rightarrow v}
\end{frame}

\begin{frame}
  Let us evaluate $(1 + 2) + (3 + 4)$.
\end{frame}

\begin{frame}
  The complete evaluation takes three steps. \\

  First:
  \begin{overprint}
    \onslide<9->
    \[\prftree[r]{E-Add1}
    {\prftree[r]{E-AddInt}
      {}
      {\highlight[white]{\strut 1 + 2} \longrightarrow \highlight[white]{\strut 3}}
    }
    {\highlight[white]{\left(\highlight[white]{\strut 1 + 2}\right) + \left(\highlight[white]{\strut 3 + 4}\right)} \longrightarrow \highlight[white]{\highlight[white]{\strut 3} + \left(\highlight[white]{3 + 4}\right)}}\]
    \onslide<1>
    \[\prftree[r]{E-Add1}
    {\prftree[r]{E-AddInt}
      {}
      {\highlight[white]{\strut 1 + 2} \longrightarrow \highlight[white]{\strut 3}}
    }
    {\highlight[hlcol3]{\left(\highlight[hlcol3]{\strut 1 + 2}\right) + \left(\highlight[hlcol3]{\strut 3 + 4}\right)} \longrightarrow \highlight[white]{\highlight[white]{\strut 3} + \left(\highlight[white]{3 + 4}\right)}}\]
    \onslide<2>
    \[\prftree[r]{E-Add1}
    {\prftree[r]{E-AddInt}
      {}
      {\highlight[white]{\strut 1 + 2} \longrightarrow \highlight[white]{\strut 3}}
    }
    {\highlight[white]{\left(\highlight[hlcol1]{\strut 1 + 2}\right) + \left(\highlight[white]{\strut 3 + 4}\right)} \longrightarrow \highlight[white]{\highlight[white]{\strut 3} + \left(\highlight[white]{3 + 4}\right)}}\]
    \onslide<3>
    \[\prftree[r]{E-Add1}
    {\prftree[r]{E-AddInt}
      {}
      {\highlight[hlcol1]{\strut 1 + 2} \longrightarrow \highlight[white]{\strut 3}}
    }
    {\highlight[white]{\left(\highlight[hlcol1]{\strut 1 + 2}\right) + \left(\highlight[white]{\strut 3 + 4}\right)} \longrightarrow \highlight[white]{\highlight[white]{\strut 3} + \left(\highlight[white]{3 + 4}\right)}}\]
    \onslide<4>
    \[\prftree[r]{E-Add1}
    {\prftree[r]{E-AddInt}
      {}
      {\highlight[hlcol1]{\strut 1 + 2} \longrightarrow \highlight[hlcol2]{\strut 3}}
    }
    {\highlight[white]{\left(\highlight[hlcol1]{\strut 1 + 2}\right) + \left(\highlight[white]{\strut 3 + 4}\right)} \longrightarrow \highlight[white]{\highlight[white]{\strut 3} + \left(\highlight[white]{3 + 4}\right)}}\]
    \onslide<5>
    \[\prftree[r]{E-Add1}
    {\prftree[r]{E-AddInt}
      {}
      {\highlight[white]{\strut 1 + 2} \longrightarrow \highlight[hlcol2]{\strut 3}}
    }
    {\highlight[white]{\left(\highlight[hlcol1]{\strut 1 + 2}\right) + \left(\highlight[white]{\strut 3 + 4}\right)} \longrightarrow \highlight[white]{\highlight[white]{\strut 3} + \left(\highlight[white]{3 + 4}\right)}}\]
    \onslide<6>
    \[\prftree[r]{E-Add1}
    {\prftree[r]{E-AddInt}
      {}
      {\highlight[white]{\strut 1 + 2} \longrightarrow \highlight[hlcol2]{\strut 3}}
    }
    {\highlight[white]{\left(\highlight[hlcol1]{\strut 1 + 2}\right) + \left(\highlight[white]{\strut 3 + 4}\right)} \longrightarrow \highlight[white]{\highlight[hlcol2]{\strut 3} + \left(\highlight[white]{3 + 4}\right)}}\]
    \onslide<7>
    \[\prftree[r]{E-Add1}
    {\prftree[r]{E-AddInt}
      {}
      {\highlight[white]{\strut 1 + 2} \longrightarrow \highlight[white]{\strut 3}}
    }
    {\highlight[white]{\left(\highlight[hlcol1]{\strut 1 + 2}\right) + \left(\highlight[white]{\strut 3 + 4}\right)} \longrightarrow \highlight[white]{\highlight[hlcol2]{\strut 3} + \left(\highlight[white]{3 + 4}\right)}}\]
    \onslide<8>
    \[\prftree[r]{E-Add1}
    {\prftree[r]{E-AddInt}
      {}
      {\highlight[white]{\strut 1 + 2} \longrightarrow \highlight[white]{\strut 3}}
    }
    {\highlight[white]{\left(\highlight[white]{\strut 1 + 2}\right) + \left(\highlight[white]{\strut 3 + 4}\right)} \longrightarrow \highlight[hlcol3]{\highlight[hlcol3]{\strut 3} + \left(\highlight[hlcol3]{3 + 4}\right)}}\]
  \end{overprint}

  Then:
  \begin{overprint}
    \onslide<1-7,16->
    \[\prftree[r]{E-Add2}
    {\prftree[r]{E-AddInt}
      {}
      {\highlight[white]{\strut 3 + 4} \longrightarrow \highlight[white]{\strut 7}}
    }
    {\highlight[white]{\highlight[white]{\strut 3} + \left(\highlight[white]{\strut 3 + 4}\right)} \longrightarrow \highlight[white]{\highlight[white]{\strut 3} + \highlight[white]{7}}}\]

    \onslide<8>
    \[\prftree[r]{E-Add2}
    {\prftree[r]{E-AddInt}
      {}
      {\highlight[white]{\strut 3 + 4} \longrightarrow \highlight[white]{\strut 7}}
    }
    {\highlight[hlcol3]{\highlight[hlcol3]{\strut 3} + \left(\highlight[hlcol3]{\strut 3 + 4}\right)} \longrightarrow \highlight[white]{\highlight[white]{\strut 3} + \highlight[white]{7}}}\]

    \onslide<9>
    \[\prftree[r]{E-Add2}
    {\prftree[r]{E-AddInt}
      {}
      {\highlight[white]{\strut 3 + 4} \longrightarrow \highlight[white]{\strut 7}}
    }
    {\highlight[white]{\highlight[white]{\strut 3} + \left(\highlight[hlcol1]{\strut 3 + 4}\right)} \longrightarrow \highlight[white]{\highlight[white]{\strut 3} + \highlight[white]{7}}}\]

    \onslide<10>
    \[\prftree[r]{E-Add2}
    {\prftree[r]{E-AddInt}
      {}
      {\highlight[hlcol1]{\strut 3 + 4} \longrightarrow \highlight[white]{\strut 7}}
    }
    {\highlight[white]{\highlight[white]{\strut 3} + \left(\highlight[hlcol1]{\strut 3 + 4}\right)} \longrightarrow \highlight[white]{\highlight[white]{\strut 3} + \highlight[white]{7}}}\]

    \onslide<11>
    \[\prftree[r]{E-Add2}
    {\prftree[r]{E-AddInt}
      {}
      {\highlight[hlcol1]{\strut 3 + 4} \longrightarrow \highlight[hlcol2]{\strut 7}}
    }
    {\highlight[white]{\highlight[white]{\strut 3} + \left(\highlight[hlcol1]{\strut 3 + 4}\right)} \longrightarrow \highlight[white]{\highlight[white]{\strut 3} + \highlight[white]{7}}}\]

    \onslide<12>
    \[\prftree[r]{E-Add2}
    {\prftree[r]{E-AddInt}
      {}
      {\highlight[white]{\strut 3 + 4} \longrightarrow \highlight[hlcol2]{\strut 7}}
    }
    {\highlight[white]{\highlight[white]{\strut 3} + \left(\highlight[hlcol1]{\strut 3 + 4}\right)} \longrightarrow \highlight[white]{\highlight[white]{\strut 3} + \highlight[white]{7}}}\]

    \onslide<13>
    \[\prftree[r]{E-Add2}
    {\prftree[r]{E-AddInt}
      {}
      {\highlight[white]{\strut 3 + 4} \longrightarrow \highlight[hlcol2]{\strut 7}}
    }
    {\highlight[white]{\highlight[white]{\strut 3} + \left(\highlight[hlcol1]{\strut 3 + 4}\right)} \longrightarrow \highlight[white]{\highlight[white]{\strut 3} + \highlight[hlcol2]{7}}}\]

    \onslide<14>
    \[\prftree[r]{E-Add2}
    {\prftree[r]{E-AddInt}
      {}
      {\highlight[white]{\strut 3 + 4} \longrightarrow \highlight[white]{\strut 7}}
    }
    {\highlight[white]{\highlight[white]{\strut 3} + \left(\highlight[hlcol1]{\strut 3 + 4}\right)} \longrightarrow \highlight[white]{\highlight[white]{\strut 3} + \highlight[hlcol2]{7}}}\]

    \onslide<15>
    \[\prftree[r]{E-Add2}
    {\prftree[r]{E-AddInt}
      {}
      {\highlight[white]{\strut 3 + 4} \longrightarrow \highlight[white]{\strut 7}}
    }
    {\highlight[white]{\highlight[white]{\strut 3} + \left(\highlight[white]{\strut 3 + 4}\right)} \longrightarrow \highlight[hlcol3]{\highlight[hlcol3]{\strut 3} + \highlight[hlcol3]{7}}}\]
  \end{overprint}

  Finally:
  \begin{overprint}
    \onslide<1-14>
    \[\prftree[r]{E-AddInt}
    {}
    {\highlight[white]{\strut 3 + 7} \longrightarrow \highlight[white]{\strut 10}}\]
    \onslide<15>
    \[\prftree[r]{E-AddInt}
    {}
    {\highlight[hlcol3]{\strut 3 + 7} \longrightarrow \highlight[white]{\strut 10}}\]
    \onslide<16>
    \[\prftree[r]{E-AddInt}
    {}
    {\highlight[hlcol1]{\strut 3 + 7} \longrightarrow \highlight[white]{\strut 10}}\]
    \onslide<17>
    \[\prftree[r]{E-AddInt}
    {}
    {\highlight[hlcol1]{\strut 3 + 7} \longrightarrow \highlight[hlcol2]{\strut 10}}\]
    \onslide<18>
    \[\prftree[r]{E-AddInt}
    {}
    {\highlight[white]{\strut 3 + 7} \longrightarrow \highlight[hlcol3]{\strut 10}}\]
  \end{overprint}
\end{frame}

\begin{frame}[fragile]
  \begin{columns}
    \begin{column}{0.5\textwidth}
  \begin{minted}{haskell}
vInt :: Rule Term ()
vInt _ (TmInt _) =
  Just ()
intValue _ _ =
  Nothing
  \end{minted}
    \end{column}
    \begin{column}{0.5\textwidth}
      \infrule[V-Int]
          {}
          {\text{value}~\left<\text{int}\right>}
    \end{column}
  \end{columns}
\end{frame}

\begin{frame}[fragile]
  \begin{minted}{haskell}
valueR :: RuleSet Term ()
valueR =
  mkRuleSet [vInt]
  \end{minted}
\end{frame}

\begin{frame}[fragile]
  \onslide<+->
  \begin{minted}{haskell}
> let tm = TmAdd (TmAdd (TmInt 1) (TmInt 2)) (TmAdd (TmInt 3) (TmInt 4))
  \end{minted}
  \onslide<+->
  \begin{minted}{haskell}
> valueR tm
  \end{minted}
  \onslide<+->
  \begin{minted}{haskell}
Nothing
  \end{minted}
  \onslide<+->
  \begin{minted}{haskell}
> valueR (TmInt 10)
  \end{minted}
  \onslide<+->
  \begin{minted}{haskell}
Just ()
  \end{minted}
\end{frame}

\begin{frame}[fragile]
  \begin{columns}
    \begin{column}{0.5\textwidth}
  \begin{minted}{haskell}
eAdd1 :: Rule Term Term
eAdd1 step (TmAdd tm1 tm2) = do
  tm1' <- step tm1
  pure $ TmAdd tm1' tm2
eAdd1 _ _ =
  Nothing
  \end{minted}
    \end{column}
    \begin{column}{0.5\textwidth}
      \infrule[E-Add1]
         {t_1 \longrightarrow {t_1}^{\prime}}
         {t_1 + t_2 \longrightarrow {t_1}^{\prime} + t_2}
    \end{column}
  \end{columns}
\end{frame}

\begin{frame}[fragile]
  \begin{columns}
    \begin{column}{0.5\textwidth}
  \begin{minted}{haskell}
eAdd2 :: RuleSet Term () -> Rule Term Term
eAdd2 value step (TmAdd tm1 tm2) = do
  _ <- value tm1
  tm2' <- step tm2
  pure $ TmAdd tm1 tm2'
eAdd2 _ _ _ =
  Nothing
  \end{minted}
    \end{column}
    \begin{column}{0.5\textwidth}
      \infrule[E-Add2]
          {t_2 \longrightarrow {t_2}^{\prime}}
          {v_1 + t_2 \longrightarrow v_1 + {t_2}^{\prime}}
    \end{column}
  \end{columns}
\end{frame}

\begin{frame}[fragile]
  \begin{columns}
    \begin{column}{0.5\textwidth}
  \begin{minted}{haskell}
eAdd2 :: Rule Term Term
eAdd2 step (TmAdd tm1@(TmInt _) tm2) = do
  tm2' <- step tm2
  pure $ TmAdd tm1 tm2'
eAdd2 _ _ =
  Nothing
  \end{minted}
    \end{column}
    \begin{column}{0.5\textwidth}
      \infrule[E-Add2]
          {t_2 \longrightarrow {t_2}^{\prime}}
          {v_1 + t_2 \longrightarrow v_1 + {t_2}^{\prime}}
    \end{column}
  \end{columns}
\end{frame}

\begin{frame}[fragile]
  \begin{columns}
    \begin{column}{0.5\textwidth}
  \begin{minted}{haskell}
eAddInt :: Rule Term Term
eAddInt _ (TmAdd (TmInt i1) (TmInt i2)) =
  Just $ TmInt (i1 + i2)
eAddInt _ _ =
  Nothing
  \end{minted}
    \end{column}
    \begin{column}{0.5\textwidth}
  \infrule[E-AddInt]
          {}
          {\left<\text{int}_1\right> + \left<\text{int}_2\right> \longrightarrow \left< \text{int}_1 + \text{int}_2 \right>}
    \end{column}
  \end{columns}
\end{frame}

\begin{frame}[fragile]
  \begin{minted}{haskell}
stepR :: RuleSet Term Term
stepR =
  mkRuleSet [eAdd1, eAdd2, eAddInt]
  \end{minted}
\end{frame}

\begin{frame}[fragile]
  \onslide<+->
  \begin{minted}{haskell}
> let tm = TmAdd (TmAdd (TmInt 1) (TmInt 2)) (TmAdd (TmInt 3) (TmInt 4))
  \end{minted}
  \onslide<+->
  \begin{minted}{haskell}
> stepR tm
  \end{minted}
  \onslide<+->
  \begin{minted}{haskell}
Just (TmAdd (TmInt 3) (TmAdd (TmInt 3) (TmInt 4)))
  \end{minted}
  \onslide<+->
  \begin{minted}{haskell}
> stepR >=> stepR $ tm
  \end{minted}
  \onslide<+->
  \begin{minted}{haskell}
Just (TmAdd (TmInt 3) (TmInt 7))
  \end{minted}
  \onslide<+->
  \begin{minted}{haskell}
> stepR >=> stepR >=> stepR $ tm
  \end{minted}
  \onslide<+->
  \begin{minted}{haskell}
Just (TmInt 10)
  \end{minted}
  \onslide<+->
  \begin{minted}{haskell}
> stepR >=> stepR >=> stepR >=> stepR $ tm
  \end{minted}
  \onslide<+->
  \begin{minted}{haskell}
Nothing
  \end{minted}
\end{frame}

\begin{frame}[fragile]
  \begin{minted}{haskell}
iterR :: RuleSet a a -> a -> a
iterR r x = case r x of
  Nothing -> x
  Just x' -> iterR r x'

eval :: Term -> Term
eval =
  iterR stepR
  \end{minted}
\end{frame}

\begin{frame}[fragile]
  \onslide<+->
  \begin{minted}{haskell}
> let tm = TmAdd (TmAdd (TmInt 1) (TmInt 2)) (TmAdd (TmInt 3) (TmInt 4))
  \end{minted}
  \onslide<+->
  \begin{minted}{haskell}
> eval tm
  \end{minted}
  \onslide<+->
  \begin{minted}{haskell}
Just (TmInt 10)
  \end{minted}
\end{frame}

\begin{frame}[c]

  {\bf Terms are not stuck}

  \bigskip

  \begin{overprint}
    \onslide<1>
    $
    \highlight[white]{\forall}
    \highlight[white]{t}
    \left (
    \highlight[white]{\text{value}~t}
    \highlight[white]{\vee}
    \highlight[white]{\exists t^{\prime} \left( t \longrightarrow t^{\prime} \right)}
    \right)
    $
    \onslide<2>
    $
    \highlight[hlcol1]{\forall}
    \highlight[white]{t}
    \left (
    \highlight[white]{\text{value}~t}
    \highlight[white]{\vee}
    \highlight[white]{\exists t^{\prime} \left( t \longrightarrow t^{\prime} \right)}
    \right)
    $
    \onslide<3>
    $
    \highlight[white]{\forall}
    \highlight[hlcol1]{t}
    \left (
    \highlight[white]{\text{value}~t}
    \highlight[white]{\vee}
    \highlight[white]{\exists t^{\prime} \left( t \longrightarrow t^{\prime} \right)}
    \right)
    $
    \onslide<4>
    $
    \highlight[white]{\forall}
    \highlight[white]{t}
    \left (
    \highlight[white]{\text{value}~t}
    \highlight[hlcol1]{\vee}
    \highlight[white]{\exists t^{\prime} \left( t \longrightarrow t^{\prime} \right)}
    \right)
    $
    \onslide<5>
    $
    \highlight[white]{\forall}
    \highlight[white]{t}
    \left (
    \highlight[hlcol1]{\text{value}~t}
    \highlight[white]{\vee}
    \highlight[white]{\exists t^{\prime} \left( t \longrightarrow t^{\prime} \right)}
    \right)
    $
    \onslide<6>
    $
    \highlight[white]{\forall}
    \highlight[white]{t}
    \left (
    \highlight[white]{\text{value}~t}
    \highlight[white]{\vee}
    \highlight[hlcol1]{\exists t^{\prime} \left( t \longrightarrow t^{\prime} \right)}
    \right)
    $
  \end{overprint}

  \bigskip

  \begin{overprint}
    \onslide<+>
    $
    \highlight[white]{\text{For all}}
    \highlight[white]{\text{terms,}} \\
    \highlight[white]{\text{the term is}}
    \highlight[white]{\text{either}}
    \text{a}
    \highlight[white]{\text{value}}
    \text{or} \\
    \highlight[white]{\text{is able to step}}
    \text{.}
    $
    \onslide<+>
    $
    \highlight[hlcol1]{\text{For all}}
    \highlight[white]{\text{terms,}} \\
    \highlight[white]{\text{the term is}}
    \highlight[white]{\text{either}}
    \text{a}
    \highlight[white]{\text{value}}
    \text{or} \\
    \highlight[white]{\text{is able to step}}
    \text{.}
    $
    \onslide<+>
    $
    \highlight[white]{\text{For all}}
    \highlight[hlcol1]{\text{terms,}} \\
    \highlight[white]{\text{the term is}}
    \highlight[white]{\text{either}}
    \text{a}
    \highlight[white]{\text{value}}
    \text{or} \\
    \highlight[white]{\text{is able to step}}
    \text{.}
    $
    \onslide<+>
    $
    \highlight[white]{\text{For all}}
    \highlight[white]{\text{terms,}} \\
    \highlight[white]{\text{the term is}}
    \highlight[hlcol1]{\text{either}}
    \text{a}
    \highlight[white]{\text{value}}
    \text{or} \\
    \highlight[white]{\text{is able to step}}
    \text{.}
    $
    \onslide<+>
    $
    \highlight[white]{\text{For all}}
    \highlight[white]{\text{terms,}} \\
    \highlight[white]{\text{the term is}}
    \highlight[white]{\text{either}}
    \text{a}
    \highlight[hlcol1]{\text{value}}
    \text{or} \\
    \highlight[white]{\text{is able to step}}
    \text{.}
    $
    \onslide<+>
    $
    \highlight[white]{\text{For all}}
    \highlight[white]{\text{terms,}} \\
    \highlight[white]{\text{the term is}}
    \highlight[white]{\text{either}}
    \text{a}
    \highlight[white]{\text{value}}
    \text{or} \\
    \highlight[hlcol1]{\text{is able to step}}
    \text{.}
    $
  \end{overprint}
\end{frame}

\begin{frame}[fragile]
  \begin{minted}{haskell}
genTerm :: Gen Term
genTerm =
  Gen.recursive Gen.choice
    [ TmInt <$> Gen.int (Range.linear 0 10) ]
    [ Gen.subterm2 genTerm genTerm TmAdd ]
  \end{minted}
\end{frame}

\begin{frame}[fragile]
  \begin{minted}{haskell}
> Gen.printTree . Gen.small $ genTerm
TmAdd (TmInt 1) (TmInt 1)
 |-TmInt 0
 |-TmInt 1
 |  |-TmInt 0
 |-TmInt 1
 |  |-TmInt 0
 |-TmAdd (TmInt 0) (TmInt 1)
 |  |-TmAdd (TmInt 0) (TmInt 0)
 |-TmAdd (TmInt 1) (TmInt 0)
  \end{minted}
\end{frame}

\begin{frame}[fragile]
  \begin{minted}{haskell}
exactlyOne :: (Show a)
           => Gen a
           -> RuleSet a b
           -> RuleSet a c
           -> Property
exactlyOne ga r1 r2 = property $ do
  a <- forAll ga
  isJust (r1 a) /== isJust (r2 a)
  \end{minted}
\end{frame}

\begin{frame}[fragile]
  \begin{minted}{haskell}
intRulesValueOrStep :: Property
intRulesExactlyOne =
  exactlyOne genTerm valueR stepR
  \end{minted}
\end{frame}

\begin{frame}[fragile]
  \onslide<1->
  \begin{minted}{haskell}
> check $ intRulesValueOrStep
  \end{minted}
  \onslide<2>
  \begin{minted}{haskell}
  <interactive> passed 100 tests.
True
  \end{minted}
\end{frame}
