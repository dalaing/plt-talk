
\section{Lambda Calculus}

\begin{frame}
  This is where we step things up a notch
\end{frame}

\begin{frame}
  \begin{mdframed}[frametitle={Terms and values}]
\begin{displaymath}
    \begin{aligned}
t \quad:=\quad& ~ \ldots &\\
  & ~ x \quad\quad &variable\\
  & ~ \lambda~x . t \quad\quad &abstraction\\
  & ~ t ~ t \quad\quad &function~application\\
v \quad:=\quad& ~ \ldots &\\
  & ~ \lambda~x . t \quad\quad &abstraction\\
    \end{aligned}
  \end{displaymath}
  \end{mdframed}
\end{frame}

\begin{frame}
  Let us look at some terms
\end{frame}

\begin{frame}
  The term
  \[x\]
  is meaningless
\end{frame}

\begin{frame}
  The term
  \[x + 2\]
  is also meaningless
\end{frame}

\begin{frame}
  The term
  \[\lambda~x~.~x + 2\]
  is an anonymous equivalent to
  \[f(x) = x + 2\]
\end{frame}

\begin{frame}
  \begin{mdframed}[frametitle={Lambda anatomy}]
    \begin{overprint}
      \onslide<1>
  \[\lambda~x~.~x + 2\]
      \onslide<2>
  \[\lambda~\highlight{x}~.~x + 2\]
      \onslide<3>
  \[\lambda~x~.~\highlight{x} + 2\]
    \end{overprint}
  \end{mdframed}
  \medskip
  \begin{overprint}
    \onslide<2>
    The $x$ to the left of the $.$ is called a variable binding.
    \onslide<3>
    The $x$ to the right of the $.$ is a variable.
  \end{overprint}
\end{frame}

\begin{frame}
  The term
  \[(\lambda~x . x + 2)~1 \]
  is equivalent to
  \[f(1)\]
  when $f$ is defined as before
\end{frame}

\begin{frame}
  In ordinary maths, we process
  \[f(1)\]
  by taking
  \[f(x) = x + 2\]
  and replacing the occurrences of $x$ with $1$ to get
  \[f(1) = 1 + 2\]
\end{frame}

\begin{frame}
  The notation for that kind of replacement is
  \[\left[ x \mapsto 1 \right] f \]
\end{frame}

\begin{frame}
  We would like to see something similar happening in our evaluation rules:
  \[(\lambda~x . x + 2) 1 \longrightarrow 1 + 2\]
\end{frame}

\begin{frame}
  \begin{mdframed}[frametitle={Small-step semantics}]
  \infrule[E-App1]
         {t_1 \longrightarrow {t_1}^{\prime}}
         {t_1~t_2 \longrightarrow {t_1}^{\prime}~t_2}
  \infrule[E-App2]
         {t_2 \longrightarrow {t_2}^{\prime}}
         {v_1~t_2 \longrightarrow v_1~{t_2}^{\prime}}
  \infrule[E-AppLam]
          {}
          {\left(\lambda~x . t_{12}\right)~v_2 \longrightarrow \left[x \mapsto v_2 \right]~t_{12}}
  \end{mdframed}
\end{frame}

\begin{frame}
  Need to be careful with substitution.
\end{frame}

\begin{frame}
  When we evaluate
  \[\left(\lambda~x . \left( \lambda~x . x + 1 \right)~\left(x + 1\right)\right)~3 \]
  we have
  \[\left[ x \mapsto 3 \right]
      \left( \left( \lambda~x . x + 1 \right)~\left(x + 1\right) \right)
  \] 
\end{frame}

\begin{frame}
  We want
  \[\left[ x \mapsto 3 \right]
      \left( \left( \lambda~x . x + 1 \right)~\left(x + 1\right) \right)
  \] 
  to become
  \[\left( \lambda~x . x + 1 \right)~(3 + 1)\]
  instead of
  \[\left( \lambda~x . 3 + 1 \right)~(3 + 1)\]
\end{frame}

\begin{frame}
  In order to do that, we need to know the {\it free variables} of
  \[
      \left( \left( \lambda~x~.~x + 1 \right)~\left(x + 1\right) \right)
  \] 
\end{frame}

\begin{frame}
  A variable is {\it bound} in a term if it appears inside a lambda abstraction with a
  matching variable binding.
  \[\lambda~x~.~\highlight{x} + 1\]
\end{frame}

\begin{frame}
  If there is no such lambda abstraction, then the variable is free in the term
  it appears in.
  \[\highlight{x} + 1\]
\end{frame}

\begin{frame}
  \begin{overprint}
    \onslide<1>
    A variable can appear as both $\highlight[white]{\text{\it free}}$ and $\highlight[white]{\text{\it bound}}$ in the same term.
  \[\left( \left( \lambda~x~.~\highlight[white]{x} + 1 \right)~\left(\highlight[white]{x} + 1\right) \right) \] 
    \onslide<2>
    A variable can appear as both $\highlight{\text{\it free}}$ and $\highlight[white]{\text{\it bound}}$ in the same term.
  \[\left( \left( \lambda~x~.~\highlight[white]{x} + 1 \right)~\left(\highlight{x} + 1\right) \right) \] 
    \onslide<3>
    A variable can appear as both $\highlight[white]{\text{\it free}}$ and $\highlight{\text{\it bound}}$ in the same term.
  \[\left( \left( \lambda~x~.~\highlight{x} + 1 \right)~\left(\highlight[white]{x} + 1\right) \right) \] 
  \end{overprint}
\end{frame}

\begin{frame}
  \begin{mdframed}[frametitle={Free variable rules}]
  \begin{overprint}
    \onslide<1>
  \begin{tabular}{L L}
     \highlight[white]{FV(x)} &= \left\{ x \right\} \\
     \highlight[white]{FV(\lambda~x.t_1)} &= FV(t_1) \setminus \left\{x\right\} \\
     \highlight[white]{FV(t_1~t_2)} &= FV(t_1) \cup FV(t_2) \\
     \\
     \highlight[white]{FV(\left<int\right>)} &= \emptyset \\
     \highlight[white]{FV(t_1 + t_2)} &= FV(t_1) \cup FV(t_2)
  \end{tabular}
    \onslide<2-3>
  \begin{tabular}{L L}
     \highlight[white]{FV(x)} &= \left\{ x \right\} \\
     \highlight[white]{FV(\lambda~x.t_1)} &= FV(t_1) \setminus \left\{x\right\} \\
     \highlight[violet]{FV(t_1~t_2)} &= FV(t_1) \cup FV(t_2) \\
     \\
     \highlight[white]{FV(\left<int\right>)} &= \emptyset \\
     \highlight[white]{FV(t_1 + t_2)} &= FV(t_1) \cup FV(t_2)
  \end{tabular}
    \onslide<4-5>
  \begin{tabular}{L L}
     \highlight[white]{FV(x)} &= \left\{ x \right\} \\
     \highlight[violet]{FV(\lambda~x.t_1)} &= FV(t_1) \setminus \left\{x\right\} \\
     \highlight[white]{FV(t_1~t_2)} &= FV(t_1) \cup FV(t_2) \\
     \\
     \highlight[white]{FV(\left<int\right>)} &= \emptyset \\
     \highlight[white]{FV(t_1 + t_2)} &= FV(t_1) \cup FV(t_2)
  \end{tabular}
    \onslide<6>
  \begin{tabular}{L L}
     \highlight[white]{FV(x)} &= \left\{ x \right\} \\
     \highlight[white]{FV(\lambda~x.t_1)} &= FV(t_1) \setminus \left\{x\right\} \\
     \highlight[white]{FV(t_1~t_2)} &= FV(t_1) \cup FV(t_2) \\
     \\
     \highlight[white]{FV(\left<int\right>)} &= \emptyset \\
     \highlight[white]{FV(t_1 + t_2)} &= FV(t_1) \cup FV(t_2)
  \end{tabular}
    \onslide<7-8>
  \begin{tabular}{L L}
     \highlight[white]{FV(x)} &= \left\{ x \right\} \\
     \highlight[white]{FV(\lambda~x.t_1)} &= FV(t_1) \setminus \left\{x\right\} \\
     \highlight[white]{FV(t_1~t_2)} &= FV(t_1) \cup FV(t_2) \\
     \\
     \highlight[white]{FV(\left<int\right>)} &= \emptyset \\
     \highlight[violet]{FV(t_1 + t_2)} &= FV(t_1) \cup FV(t_2)
  \end{tabular}
    \onslide<9-10>
  \begin{tabular}{L L}
     \highlight[violet]{FV(x)} &= \left\{ x \right\} \\
     \highlight[white]{FV(\lambda~x.t_1)} &= FV(t_1) \setminus \left\{x\right\} \\
     \highlight[white]{FV(t_1~t_2)} &= FV(t_1) \cup FV(t_2) \\
     \\
     \highlight[white]{FV(\left<int\right>)} &= \emptyset \\
     \highlight[white]{FV(t_1 + t_2)} &= FV(t_1) \cup FV(t_2)
  \end{tabular}
    \onslide<11-12>
  \begin{tabular}{L L}
     \highlight[white]{FV(x)} &= \left\{ x \right\} \\
     \highlight[white]{FV(\lambda~x.t_1)} &= FV(t_1) \setminus \left\{x\right\} \\
     \highlight[white]{FV(t_1~t_2)} &= FV(t_1) \cup FV(t_2) \\
     \\
     \highlight[violet]{FV(\left<int\right>)} &= \emptyset \\
     \highlight[white]{FV(t_1 + t_2)} &= FV(t_1) \cup FV(t_2)
  \end{tabular}
    \onslide<13->
  \begin{tabular}{L L}
     \highlight[white]{FV(x)} &= \left\{ x \right\} \\
     \highlight[white]{FV(\lambda~x.t_1)} &= FV(t_1) \setminus \left\{x\right\} \\
     \highlight[white]{FV(t_1~t_2)} &= FV(t_1) \cup FV(t_2) \\
     \\
     \highlight[white]{FV(\left<int\right>)} &= \emptyset \\
     \highlight[white]{FV(t_1 + t_2)} &= FV(t_1) \cup FV(t_2)
  \end{tabular}
  \end{overprint}
  \end{mdframed}
  \medskip
  \begin{overprint}
    \onslide<1-2>
    \begin{tabular}{L L}
    FV(\left( \lambda~x . x + 1 \right)~\left(x + 1\right)) &= ~ ?
    \end{tabular}
    \onslide<3-4>
    \begin{tabular}{L L}
    FV(\left( \lambda~x . x + 1 \right)~\left(x + 1\right)) &= FV(\lambda~x . x +
    1) \cup FV(x + 1)
    \end{tabular}
    \onslide<5>
    \begin{tabular}{L L}
    FV(\left( \lambda~x . x + 1 \right)~\left(x + 1\right)) &= \left( FV(x + 1)
      \setminus \left\{ x \right\} \right) \cup FV(x + 1)
    \end{tabular}
    \onslide<6-7>
    \begin{tabular}{L L}
    FV(\left( \lambda~x . x + 1 \right)~\left(x + 1\right)) &= ( \highlight[white]{FV(x + 1)} \setminus \left\{ x \right\} ) \cup \highlight[white]{FV(x + 1)} \\
    FV(x + 1) &= ~ ?
    \end{tabular}
    \onslide<8-9>
    \begin{tabular}{L L}
    FV(\left( \lambda~x . x + 1 \right)~\left(x + 1\right)) &= ( \highlight[white]{FV(x + 1)} \setminus \left\{ x \right\} ) \cup \highlight[white]{FV(x + 1)} \\
    FV(x + 1) &= FV(x) \cup FV(1)
    \end{tabular}
    \onslide<10-11>
    \begin{tabular}{L L}
    FV(\left( \lambda~x . x + 1 \right)~\left(x + 1\right)) &= ( \highlight[white]{FV(x + 1)} \setminus \left\{ x \right\} ) \cup \highlight[white]{FV(x + 1)} \\
    FV(x + 1) &= \left\{ x \right\} \cup FV(1)
    \end{tabular}
    \onslide<12>
    \begin{tabular}{L L}
    FV(\left( \lambda~x . x + 1 \right)~\left(x + 1\right)) &= ( \highlight[white]{FV(x + 1)} \setminus \left\{ x \right\} ) \cup \highlight[white]{FV(x + 1)} \\
     FV(x + 1) &= \left\{ x \right\} \cup \emptyset
    \end{tabular}
    \onslide<13>
    \begin{tabular}{L L}
    FV(\left( \lambda~x . x + 1 \right)~\left(x + 1\right)) &= ( \highlight[white]{FV(x + 1)} \setminus \left\{ x \right\} ) \cup \highlight[white]{FV(x + 1)} \\
     FV(x + 1) &= \left\{ x \right\}
    \end{tabular}
    \onslide<14>
    \begin{tabular}{L L}
    FV(\left( \lambda~x . x + 1 \right)~\left(x + 1\right)) &= ( \highlight[violet]{FV(x + 1)} \setminus \left\{ x \right\} ) \cup \highlight[white]{FV(x + 1)} \\
     FV(x + 1) &= \left\{ x \right\}
    \end{tabular}
    \onslide<15>
    \begin{tabular}{L L}
     FV(\left( \lambda~x . x + 1 \right)~\left(x + 1\right)) &= \left( \left\{ x \right\} \setminus \left\{ x \right\} \right) \cup \highlight[white]{FV(x + 1)} \\
     FV(x + 1) &= \left\{ x \right\}
    \end{tabular}
    \onslide<16>
    \begin{tabular}{L L}
    FV(\left( \lambda~x . x + 1 \right)~\left(x + 1\right)) &= \emptyset \cup \highlight[white]{FV(x + 1)} \\
     FV(x + 1) &= \left\{ x \right\}
    \end{tabular}
    \onslide<17>
    \begin{tabular}{L L}
     FV(\left( \lambda~x . x + 1 \right)~\left(x + 1\right)) &= \highlight[white]{FV(x + 1)} \\
     FV(x + 1) &= \left\{ x \right\}
    \end{tabular}
    \onslide<18>
    \begin{tabular}{L L}
     FV(\left( \lambda~x . x + 1 \right)~\left(x + 1\right)) &= \highlight[violet]{FV(x + 1)} \\
     FV(x + 1) &= \left\{ x \right\}
    \end{tabular}
    \onslide<19>
    \begin{tabular}{L L}
     FV(\left( \lambda~x . x + 1 \right)~\left(x + 1\right)) &= \left\{ x \right\} \\
     FV(x + 1) &= \left\{ x \right\}
    \end{tabular}
  \end{overprint}
\end{frame}

\begin{frame}
  \begin{mdframed}[frametitle={Substitution rules}]
  \begin{overprint}
    \onslide<1>
  \begin{tabular}{L L L}
    \highlight[white]{\left[ x \mapsto s \right] x} &= s & \\
    \highlight[white]{\left[ x \mapsto s \right] y} &= y & \text{if}~y\neq x \\
    \highlight[white]{\left[ x \mapsto s \right] \left( \lambda y . t_1 \right)} &= \lambda y . \left( \left[ x \mapsto s \right] t_1 \right) & \text{if}~y\neq x \wedge y \notin FV(s) \\
    \highlight[white]{\left[ x \mapsto s \right] \left(t_1~t_2\right)} &= \left(\left[x \mapsto s  \right] t_1 \right)~\left(\left[x \mapsto s \right] t_2 \right) & \\
    \\
    \highlight[white]{\left[x \mapsto s\right]\left<int\right>} &= \left<int\right>& \\
    \highlight[white]{\left[ x \mapsto s \right] \left(t_1 + t_2\right)} &= \left(\left[x \mapsto s  \right] t_1 \right) + \left(\left[x \mapsto s \right] t_2 \right) &
  \end{tabular}
    \onslide<2-3>
  \begin{tabular}{L L L}
    \highlight[white]{\left[ x \mapsto s \right] x} &= s & \\
    \highlight[white]{\left[ x \mapsto s \right] y} &= y & \text{if}~y\neq x \\
    \highlight[white]{\left[ x \mapsto s \right] \left( \lambda y . t_1 \right)} &= \lambda y . \left( \left[ x \mapsto s \right] t_1 \right) & \text{if}~y\neq x \wedge y \notin FV(s) \\
    \highlight[violet]{\left[ x \mapsto s \right] \left(t_1~t_2\right)} &= \left(\left[x \mapsto s  \right] t_1 \right)~\left(\left[x \mapsto s \right] t_2 \right) & \\
    \\
    \highlight[white]{\left[x \mapsto s\right]\left<int\right>} &= \left<int\right>& \\
    \highlight[white]{\left[ x \mapsto s \right] \left(t_1 + t_2\right)} &= \left(\left[x \mapsto s  \right] t_1 \right) + \left(\left[x \mapsto s \right] t_2 \right) &
  \end{tabular}
    \onslide<4-6>
  \begin{tabular}{L L L}
    \highlight[white]{\left[ x \mapsto s \right] x} &= s & \\
    \highlight[white]{\left[ x \mapsto s \right] y} &= y & \text{if}~y\neq x \\
    \highlight[violet]{\left[ x \mapsto s \right] \left( \lambda y . t_1 \right)} &= \lambda y . \left( \left[ x \mapsto s \right] t_1 \right) & \text{if}~y\neq x \wedge y \notin FV(s) \\
    \highlight[white]{\left[ x \mapsto s \right] \left(t_1~t_2\right)} &= \left(\left[x \mapsto s  \right] t_1 \right)~\left(\left[x \mapsto s \right] t_2 \right) & \\
    \\
    \highlight[white]{\left[x \mapsto s\right]\left<int\right>} &= \left<int\right>& \\
    \highlight[white]{\left[ x \mapsto s \right] \left(t_1 + t_2\right)} &= \left(\left[x \mapsto s  \right] t_1 \right) + \left(\left[x \mapsto s \right] t_2 \right) &
  \end{tabular}
    \onslide<7-8>
  \begin{tabular}{L L L}
    \highlight[white]{\left[ x \mapsto s \right] x} &= s & \\
    \highlight[white]{\left[ x \mapsto s \right] y} &= y & \text{if}~y\neq x \\
    \highlight[white]{\left[ x \mapsto s \right] \left( \lambda y . t_1 \right)} &= \lambda y . \left( \left[ x \mapsto s \right] t_1 \right) & \text{if}~y\neq x \wedge y \notin FV(s) \\
    \highlight[white]{\left[ x \mapsto s \right] \left(t_1~t_2\right)} &= \left(\left[x \mapsto s  \right] t_1 \right)~\left(\left[x \mapsto s \right] t_2 \right) & \\
    \\
    \highlight[white]{\left[x \mapsto s\right]\left<int\right>} &= \left<int\right>& \\
    \highlight[violet]{\left[ x \mapsto s \right] \left(t_1 + t_2\right)} &= \left(\left[x \mapsto s  \right] t_1 \right) + \left(\left[x \mapsto s \right] t_2 \right) &
  \end{tabular}
    \onslide<9-10>
  \begin{tabular}{L L L}
    \highlight[white]{\left[ x \mapsto s \right] x} &= s & \\
    \highlight[violet]{\left[ x \mapsto s \right] y} &= y & \text{if}~y\neq x \\
    \highlight[white]{\left[ x \mapsto s \right] \left( \lambda y . t_1 \right)} &= \lambda y . \left( \left[ x \mapsto s \right] t_1 \right) & \text{if}~y\neq x \wedge y \notin FV(s) \\
    \highlight[white]{\left[ x \mapsto s \right] \left(t_1~t_2\right)} &= \left(\left[x \mapsto s  \right] t_1 \right)~\left(\left[x \mapsto s \right] t_2 \right) & \\
    \\
    \highlight[white]{\left[x \mapsto s\right]\left<int\right>} &= \left<int\right>& \\
    \highlight[white]{\left[ x \mapsto s \right] \left(t_1 + t_2\right)} &= \left(\left[x \mapsto s  \right] t_1 \right) + \left(\left[x \mapsto s \right] t_2 \right) &
  \end{tabular}
    \onslide<11-12>
  \begin{tabular}{L L L}
    \highlight[white]{\left[ x \mapsto s \right] x} &= s & \\
    \highlight[white]{\left[ x \mapsto s \right] y} &= y & \text{if}~y\neq x \\
    \highlight[white]{\left[ x \mapsto s \right] \left( \lambda y . t_1 \right)} &= \lambda y . \left( \left[ x \mapsto s \right] t_1 \right) & \text{if}~y\neq x \wedge y \notin FV(s) \\
    \highlight[white]{\left[ x \mapsto s \right] \left(t_1~t_2\right)} &= \left(\left[x \mapsto s  \right] t_1 \right)~\left(\left[x \mapsto s \right] t_2 \right) & \\
    \\
    \highlight[violet]{\left[x \mapsto s\right]\left<int\right>} &= \left<int\right>& \\
    \highlight[white]{\left[ x \mapsto s \right] \left(t_1 + t_2\right)} &= \left(\left[x \mapsto s  \right] t_1 \right) + \left(\left[x \mapsto s \right] t_2 \right) &
  \end{tabular}
    \onslide<13-14>
  \begin{tabular}{L L L}
    \highlight[white]{\left[ x \mapsto s \right] x} &= s & \\
    \highlight[white]{\left[ x \mapsto s \right] y} &= y & \text{if}~y\neq x \\
    \highlight[white]{\left[ x \mapsto s \right] \left( \lambda y . t_1 \right)} &= \lambda y . \left( \left[ x \mapsto s \right] t_1 \right) & \text{if}~y\neq x \wedge y \notin FV(s) \\
    \highlight[white]{\left[ x \mapsto s \right] \left(t_1~t_2\right)} &= \left(\left[x \mapsto s  \right] t_1 \right)~\left(\left[x \mapsto s \right] t_2 \right) & \\
    \\
    \highlight[white]{\left[x \mapsto s\right]\left<int\right>} &= \left<int\right>& \\
    \highlight[violet]{\left[ x \mapsto s \right] \left(t_1 + t_2\right)} &= \left(\left[x \mapsto s  \right] t_1 \right) + \left(\left[x \mapsto s \right] t_2 \right) &
  \end{tabular}
    \onslide<15-16>
  \begin{tabular}{L L L}
    \highlight[violet]{\left[ x \mapsto s \right] x} &= s & \\
    \highlight[white]{\left[ x \mapsto s \right] y} &= y & \text{if}~y\neq x \\
    \highlight[white]{\left[ x \mapsto s \right] \left( \lambda y . t_1 \right)} &= \lambda y . \left( \left[ x \mapsto s \right] t_1 \right) & \text{if}~y\neq x \wedge y \notin FV(s) \\
    \highlight[white]{\left[ x \mapsto s \right] \left(t_1~t_2\right)} &= \left(\left[x \mapsto s  \right] t_1 \right)~\left(\left[x \mapsto s \right] t_2 \right) & \\
    \\
    \highlight[white]{\left[x \mapsto s\right]\left<int\right>} &= \left<int\right>& \\
    \highlight[white]{\left[ x \mapsto s \right] \left(t_1 + t_2\right)} &= \left(\left[x \mapsto s  \right] t_1 \right) + \left(\left[x \mapsto s \right] t_2 \right) &
  \end{tabular}
    \onslide<17-18>
  \begin{tabular}{L L L}
    \highlight[white]{\left[ x \mapsto s \right] x} &= s & \\
    \highlight[white]{\left[ x \mapsto s \right] y} &= y & \text{if}~y\neq x \\
    \highlight[white]{\left[ x \mapsto s \right] \left( \lambda y . t_1 \right)} &= \lambda y . \left( \left[ x \mapsto s \right] t_1 \right) & \text{if}~y\neq x \wedge y \notin FV(s) \\
    \highlight[white]{\left[ x \mapsto s \right] \left(t_1~t_2\right)} &= \left(\left[x \mapsto s  \right] t_1 \right)~\left(\left[x \mapsto s \right] t_2 \right) & \\
    \\
    \highlight[violet]{\left[x \mapsto s\right]\left<int\right>} &= \left<int\right>& \\
    \highlight[white]{\left[ x \mapsto s \right] \left(t_1 + t_2\right)} &= \left(\left[x \mapsto s  \right] t_1 \right) + \left(\left[x \mapsto s \right] t_2 \right) &
  \end{tabular}
  \end{overprint}
  \end{mdframed}

  \medskip

  \begin{overprint}
    \onslide<1-2>
  \begin{tabular}{L}
    \left[ x \mapsto 3 \right] \left( \left( \lambda~x . x + 1 \right)~\left(x + 1\right) \right) =
     ?
  \end{tabular}
    \onslide<3-4>
  \begin{tabular}{L}
    \left[ x \mapsto 3 \right] \left( \left( \lambda~x . x + 1 \right)~\left(x + 1\right) \right) =
    \left( \left[ x \mapsto 3 \right]\left(\lambda x . x + 1\right)\right)~\left(\left[ x \mapsto 3 \right]\left(x + 1\right)\right)
  \end{tabular}
    \onslide<5>
  \begin{tabular}{L}
    \left[ x \mapsto 3 \right] \left( \left( \lambda~x . x + 1 \right)~\left(x + 1\right) \right) =
    \left( \left[ x \mapsto 3 \right]\left(\lambda z . z + 1\right)\right)~\left(\left[ x \mapsto 3 \right]\left(x + 1\right)\right)
  \end{tabular}
    \onslide<6-7>
  \begin{tabular}{L}
    \left[ x \mapsto 3 \right] \left( \left( \lambda~x . x + 1 \right)~\left(x + 1\right) \right) =
    \left(\lambda z . \left[x \mapsto 3 \right]\left(z + 1\right)\right)~\left(\left[ x \mapsto 3 \right]\left(x + 1\right)\right)
  \end{tabular}
    \onslide<8-9>
  \begin{tabular}{L}
    \left[ x \mapsto 3 \right] \left( \left( \lambda~x . x + 1 \right)~\left(x + 1\right) \right) =
    \left(\lambda z . \left(\left[x \mapsto 3 \right]z + \left[x \mapsto 3 \right]1\right)\right)~\left(\left[ x \mapsto 3 \right]\left(x + 1\right)\right)
  \end{tabular}
    \onslide<10-11>
  \begin{tabular}{L}
    \left[ x \mapsto 3 \right] \left( \left( \lambda~x . x + 1 \right)~\left(x + 1\right) \right) =
    \left(\lambda z . \left(z + \left[x \mapsto 3 \right]1\right)\right)~\left(\left[ x \mapsto 3 \right]\left(x + 1\right)\right)
  \end{tabular}
    \onslide<12-13>
  \begin{tabular}{L}
    \left[ x \mapsto 3 \right] \left( \left( \lambda~x . x + 1 \right)~\left(x + 1\right) \right) =
    \left(\lambda z . z + 1\right)~\left(\left[ x \mapsto 3 \right]\left(x + 1\right)\right)
  \end{tabular}
    \onslide<14-15>
  \begin{tabular}{L}
    \left[ x \mapsto 3 \right] \left( \left( \lambda~x . x + 1 \right)~\left(x + 1\right) \right) =
    \left(\lambda z . z + 1\right)~\left(\left[ x \mapsto 3 \right]x + \left[x \mapsto 3\right]1\right)
  \end{tabular}
    \onslide<16-17>
  \begin{tabular}{L}
    \left[ x \mapsto 3 \right] \left( \left( \lambda~x . x + 1 \right)~\left(x + 1\right) \right) =
    \left(\lambda z . z + 1\right)~\left(3 + \left[x \mapsto 3\right]1\right)
  \end{tabular}
    \onslide<18>
  \begin{tabular}{L}
    \left[ x \mapsto 3 \right] \left( \left( \lambda~x . x + 1 \right)~\left(x + 1\right) \right) =
    \left(\lambda z . z + 1\right)~\left(3 + 1\right)
  \end{tabular}
  \end{overprint}
\end{frame}

\begin{frame}
  What can we do with lambda calculus?
\end{frame}

\begin{frame}
  We can do Booleans:
  \begin{align*}
   tru &= \lambda~t.~\lambda~f.~t \\
   fls &= \lambda~t.~\lambda~f.~f \\
   and &= \lambda~b.~\lambda~c.~b~c~fls
  \end{align*}
\end{frame}

\begin{frame}
  We can do natural numbers:
  \begin{align*}
  z &= \lambda~s.~\lambda~z.~z \\
  scc &= \lambda~n.~\lambda~s.~\lambda~z.~s~\left(n~s~z\right) \\
  plus~m~n &= \lambda~s.~\lambda~z.~m~s~\left(n~s~z\right)
  \end{align*}
\end{frame}

\begin{frame}
  We can do pairs:
  \begin{align*}
    pair &= \lambda~f.~\lambda~s.~\lambda~b.~b~f~s \\
    fst &= \lambda~p.~p~tru \\
    snd &= \lambda~p.~p~fls \\
  \end{align*}
  \[fst~\left(pair~v~w\right) \Rightarrow v\]
\end{frame}

\begin{frame}
  We even have enough to do recursion:
  
  \[fix = \lambda~f.~\left(\lambda~x.~f~\left(\lambda~y.~x~x~y\right)\right)~\left(\lambda~x.~f~\left(\lambda~y.~x~x~y\right)\right)\]

  \[g = \lambda~fct.~\lambda~n.~if~eq~n~0~then~1~else~times~n~\left(fct~prd~n\right)\]

  \[factorial = fix~g\]
\end{frame}

\begin{frame}
  Sometimes those kind of hijinx lead us into trouble:
  
  \[omega = \left(\lambda~x.~x~x\right) \left(\lambda~x.~x~x\right)\]

  \[omega \Rightarrow omega\]
\end{frame}

\begin{frame}
  One other big problem - there are plenty of stuck terms:
  \[1~2\]
\end{frame}


