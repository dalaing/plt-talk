
\section{Booleans and Natural numbers}

\begin{frame}[c]
  We are going to combine a few of these languages.
\end{frame}

\begin{frame}
  \begin{mdframed}[frametitle={Terms}]
\begin{displaymath}
    \begin{aligned}
t \quad:=\quad& ~ &\\
  & ~ \text{false} \quad\quad &constant~false\\
  & ~ \text{true} \quad\quad &constant~true\\
  & ~ t ~ \text{or} ~ t \quad\quad &disjunction\\
  & ~ \text{O} \quad\quad &constant~zero \\
  & ~ \text{succ}~t \quad\quad &successor \\
  & ~ \text{pred}~t \quad\quad &predecessor \\
  & ~ \highlight{\text{iszero}~t} \quad\quad &iszero \\
  & ~ \highlight{\text{if}~t~\text{then}~t~\text{else}~t} \quad\quad &if \\
    \end{aligned}
  \end{displaymath}
  \end{mdframed}
\end{frame}

\begin{frame}
  \begin{mdframed}[frametitle={Small-step semantics for $\text{iszero}$}]
  \infrule[E-IsZero]
         {t \longrightarrow t^{\prime}}
         {\text{iszero}~t \longrightarrow \text{iszero}~t^{\prime}}
  \infrule[E-IsZeroZero]
         {}
         {\text{iszero}~\text{O} \longrightarrow \text{true}}
  \infrule[$\text{E-IsZeroSucc}^{\text{*}}$]
        {}
        {\text{iszero}~\left( \text{succ}~v \right) \longrightarrow \text{false}}
  \end{mdframed}
  \medskip
  * Uses $\text{t}$ in place of $\text{v}$ for the lazy version.
\end{frame}

\begin{frame}
  \begin{mdframed}[frametitle={Small-step semantics for $\text{if}$}]
  \infrule[E-If]
         {t_1 \longrightarrow {t_1}^{\prime}}
         {
           \text{if}~t_1~\text{then}~t_2~\text{else}~t_3
           \longrightarrow
           \text{if}~{t_1}^{\prime}~\text{then}~t_2~\text{else}~t_3
        }
  \infrule[E-IfTrue]
          {}       
          {
           \text{if}~\text{true}~\text{then}~t_2~\text{else}~t_3
           \longrightarrow
           t_2
          }       
  \infrule[E-IfFalse]
          {}       
          {
           \text{if}~\text{false}~\text{then}~t_2~\text{else}~t_3
           \longrightarrow
           t_3
          }       
  \end{mdframed}
\end{frame}

\begin{frame}
  This term is stuck:
\begin{displaymath}
\text{iszero}~\text{false}
\end{displaymath}
 as is this one:
\begin{displaymath}
\text{if}~\text{O}~\text{then}~\text{false}~\text{else}~\text{true}
\end{displaymath}
\end{frame}

\begin{frame}[c]
  We can break values down further to try to keep things on-track.
\end{frame}

\begin{frame}
  \begin{mdframed}[frametitle={Values}]
\begin{displaymath}
    \begin{aligned}
bv \quad:=\quad& ~ &\\
  & ~ \text{false} \quad\quad &false~value \\
  & ~ \text{true} \quad\quad &true~value \\
nv \quad:=\quad& ~ \qquad\qquad &\\
  & ~ \text{O} \quad\quad &zero~value \\
  & ~ \text{succ}~nv \quad\quad &successor~value \\
v \quad:=\quad& ~ \qquad\qquad &\\
  & ~ bv \quad\quad &boolean~value \\
  & ~ nv \quad\quad &natural~number~value \\
    \end{aligned}
  \end{displaymath}
  \end{mdframed}
\end{frame}

\begin{frame}
  \begin{mdframed}[frametitle={Small-step semantics}]
  \infrule[E-IsZeroSucc]
        {}
        {\text{iszero}~ \left( \text{succ}~\highlight{nv} \right) \longrightarrow \text{false}}
  \end{mdframed}
\end{frame}

\begin{frame}
  These terms are still stuck:
\begin{displaymath}
\text{iszero}~\text{false}
\end{displaymath}
\begin{displaymath}
\text{if}~\text{O}~\text{then}~\text{false}~\text{else}~\text{true}
\end{displaymath}
\end{frame}

\begin{frame}[c]
  The finer-grained values have more clearly communicated intent.
\end{frame}

\begin{frame}[c]
  The more detailed break down of values may effect {\it when} a term gets stuck, but not whether a term will get stuck.
\end{frame}

\begin{frame}[c]
  We want to work out which terms will or won't get stuck without having to evaluate the terms.
\end{frame}

\begin{frame}[c]
  Enter {\it types}.
\end{frame}

\begin{frame}
  \begin{mdframed}[frametitle={Types}]
\begin{displaymath}
    \begin{aligned}
\text{T} \quad:=\quad& ~ &\\
  & ~ \text{Bool} \quad\quad &\text{type of booleans} \\
  & ~ \text{Nat} \quad\quad &\text{type of natural numbers} \\
    \end{aligned}
  \end{displaymath}
  \end{mdframed}
\end{frame}

\begin{frame}

  \begin{mdframed}[frametitle={Typing rules (Booleans)}]

  \infrule[T-False]
          {}
          {\vdash \text{false} {:} \text{Bool}}
  \infrule[T-True]
          {}
          {\vdash \text{true} {:} \text{Bool}}
  \infrule[T-Or]
          {\vdash t_1 {:} \text{Bool} \andalso \vdash t_2 {:} \text{Bool}}
          {\vdash t_1~\text{or}~t_2 {:} \text{Bool}}
  \end{mdframed}

  \medskip
  
  \begin{overprint}
    \onslide<+>
  We have a binary relation $\vdash t {:} T$.
    \onslide<+>
  This indicates that the term $t$ is {\it well-typed} and has type $T$.
    \onslide<+>
  Any term which doesn't match any of these rules is {\it ill-typed}.
    \onslide<+>
  Anything on the left of the $\vdash$ is additional context that the typing rules need.
    \onslide<+>
  For now we don't need any more context.
    \onslide<+>
  We can use these rules to check that a given term has a particular type.
    \onslide<+>
  We can use these rules to {\it infer} the type for a particular term.
    \onslide<+>
    At this point our type inference is syntax-directed - we can walk through
    the syntax tree, applying one rule at a time.
  \end{overprint}
  
\end{frame}

\begin{frame}

  \begin{mdframed}[frametitle={Typing rules (Natural numbers)}]

  \infrule[T-Zero]
          {}
          {\vdash \text{O} {:} \text{Nat}}
  \infrule[T-Succ]
          {\vdash t {:} \text{Nat}}
          {\vdash \text{succ}~t {:} \text{Nat}}
  \infrule[T-Pred]
          {\vdash t {:} \text{Nat}}
          {\vdash \text{pred}~t {:} \text{Nat}}

  \end{mdframed}
\end{frame}

\begin{frame}

  \begin{mdframed}[frametitle={Typing rules (both)}]
  
  \infrule[T-IsZero]
          {\vdash t {:} \text{Nat}}
          {\vdash \text{iszero}~t {:} \text{Bool}}
  \infrule[T-If]
          {\vdash t_1 {:} \text{Bool} \andalso \vdash t_2 {:} T \andalso \vdash t_3 {:} T}
          {\vdash \text{if}~t_1~\text{then}~t_2~\text{else}~t_3 {:} T}

  \end{mdframed}

  \medskip
  
  \begin{overprint}
    \onslide<+>
    \onslide<+>
  These rule out the stuck terms we saw previously:

  \begin{displaymath}
  \text{iszero}~\text{false}
  \end{displaymath}
  and
  \begin{displaymath}
  \text{if}~\text{O}~\text{then}~\text{false}~\text{else}~\text{true}
  \end{displaymath}
    \onslide<+>
    They also rule out terms that are not stuck:
  \begin{displaymath}
  \text{if}~\text{true}~\text{then}~\text{O}~\text{else}~\text{true}
  \end{displaymath}
    as the rule $\text{T-If}$ states that both branches of the $\text{if}$ have to have the same type.
    \onslide<+>
    This kind of thing is normally not a big deal.
    \onslide<+>
    When a type system rules out some terms that were not stuck, it is called a {\it conservative} type system.
  \end{overprint}

\end{frame}

\begin{frame}[c]
  For now, each well-typed term will have a unique type.
\end{frame}

\begin{frame}[c]
  Later on that will relax, and we'll be more concerned with the {\it principal
    type} of a term.
\end{frame}

\begin{frame}[c]
  There are two main properties which relate the type system of a language and
  the small-step semantics of a language.
\end{frame}

\begin{frame}[c]

  {\bf Progress}

  \bigskip

  \begin{overprint}
    \onslide<1>
    $
    \highlight[white]{\forall}
    \highlight[white]{\vdash t {:} T}
    \left (
    \highlight[white]{\text{value}~t}
    \highlight[white]{\vee}
    \highlight[white]{\exists t^{\prime} \left( t \longrightarrow t^{\prime} \right)}
    \right)
    $
    \onslide<2>
    $
    \highlight{\forall}
    \highlight[white]{\vdash t {:} T}
    \left (
    \highlight[white]{\text{value}~t}
    \highlight[white]{\vee}
    \highlight[white]{\exists t^{\prime} \left( t \longrightarrow t^{\prime} \right)}
    \right)
    $
    \onslide<3>
    $
    \highlight[white]{\forall}
    \highlight{\vdash t {:} T}
    \left (
    \highlight[white]{\text{value}~t}
    \highlight[white]{\vee}
    \highlight[white]{\exists t^{\prime} \left( t \longrightarrow t^{\prime} \right)}
    \right)
    $
    \onslide<4>
    $
    \highlight[white]{\forall}
    \highlight[white]{\vdash t {:} T}
    \left (
    \highlight[white]{\text{value}~t}
    \highlight{\vee}
    \highlight[white]{\exists t^{\prime} \left( t \longrightarrow t^{\prime} \right)}
    \right)
    $
    \onslide<5>
    $
    \highlight[white]{\forall}
    \highlight[white]{\vdash t {:} T}
    \left (
    \highlight{\text{value}~t}
    \highlight[white]{\vee}
    \highlight[white]{\exists t^{\prime} \left( t \longrightarrow t^{\prime} \right)}
    \right)
    $
    \onslide<6>
    $
    \highlight[white]{\forall}
    \highlight[white]{\vdash t {:} T}
    \left (
    \highlight[white]{\text{value}~t}
    \highlight[white]{\vee}
    \highlight{\exists t^{\prime} \left( t \longrightarrow t^{\prime} \right)}
    \right)
    $
  \end{overprint}

  \bigskip

  \begin{overprint}
    \onslide<+>
    $
    \highlight[white]{\text{For all}}
    \highlight[white]{\text{well-typed terms,}} \\
    \highlight[white]{\text{the term is}}
    \highlight[white]{\text{either}}
    \text{a}
    \highlight[white]{\text{value}}
    \text{or} \\
    \highlight[white]{\text{is able to step}}
    \text{.}
    $
    \onslide<+>
    $
    \highlight{\text{For all}}
    \highlight[white]{\text{well-typed terms,}} \\
    \highlight[white]{\text{the term is}}
    \highlight[white]{\text{either}}
    \text{a}
    \highlight[white]{\text{value}}
    \text{or} \\
    \highlight[white]{\text{is able to step}}
    \text{.}
    $
    \onslide<+>
    $
    \highlight[white]{\text{For all}}
    \highlight{\text{well-typed terms,}} \\
    \highlight[white]{\text{the term is}}
    \highlight[white]{\text{either}}
    \text{a}
    \highlight[white]{\text{value}}
    \text{or} \\
    \highlight[white]{\text{is able to step}}
    \text{.}
    $
    \onslide<+>
    $
    \highlight[white]{\text{For all}}
    \highlight[white]{\text{well-typed terms,}} \\
    \highlight[white]{\text{the term is}}
    \highlight{\text{either}}
    \text{a}
    \highlight[white]{\text{value}}
    \text{or} \\
    \highlight[white]{\text{is able to step}}
    \text{.}
    $
    \onslide<+>
    $
    \highlight[white]{\text{For all}}
    \highlight[white]{\text{well-typed terms,}} \\
    \highlight[white]{\text{the term is}}
    \highlight[white]{\text{either}}
    \text{a}
    \highlight{\text{value}}
    \text{or} \\
    \highlight[white]{\text{is able to step}}
    \text{.}
    $
    \onslide<+>
    $
    \highlight[white]{\text{For all}}
    \highlight[white]{\text{well-typed terms,}} \\
    \highlight[white]{\text{the term is}}
    \highlight[white]{\text{either}}
    \text{a}
    \highlight[white]{\text{value}}
    \text{or} \\
    \highlight{\text{is able to step}}
    \text{.}
    $
  \end{overprint}
\end{frame}

\begin{frame}[c]

  {\bf Preservation}

  \bigskip

  \begin{overprint}
    \onslide<1>
    $
    \highlight[white]{\forall}
    \highlight[white]{\vdash t {:} T}
    \left(
      \highlight[white]{\exists t^{\prime} \left(t \longrightarrow t^{\prime}\right)}
      \highlight[white]{\implies}
      \highlight[white]{\vdash t^{\prime} {:} T}
    \right)
    $
    \onslide<2>
    $
    \highlight{\forall}
    \highlight[white]{\vdash t {:} T}
    \left(
      \highlight[white]{\exists t^{\prime} \left(t \longrightarrow t^{\prime}\right)}
      \highlight[white]{\implies}
      \highlight[white]{\vdash t^{\prime} {:} T}
    \right)
    $
    \onslide<3>
    $
    \highlight[white]{\forall}
    \highlight{\vdash t {:} T}
    \left(
      \highlight[white]{\exists t^{\prime} \left(t \longrightarrow t^{\prime}\right)}
      \highlight[white]{\implies}
      \highlight[white]{\vdash t^{\prime} {:} T}
    \right)
    $
    \onslide<4>
    $
    \highlight[white]{\forall}
    \highlight[white]{\vdash t {:} T}
    \left(
      \highlight[white]{\exists t^{\prime} \left(t \longrightarrow t^{\prime}\right)}
      \highlight{\implies}
      \highlight[white]{\vdash t^{\prime} {:} T}
    \right)
    $
    \onslide<5>
    $
    \highlight[white]{\forall}
    \highlight[white]{\vdash t {:} T}
    \left(
      \highlight{\exists t^{\prime} \left(t \longrightarrow t^{\prime}\right)}
      \highlight[white]{\implies}
      \highlight[white]{\vdash t^{\prime} {:} T}
    \right)
    $
    \onslide<6>
    $
    \highlight[white]{\forall}
    \highlight{\vdash t {:} T}
    \left(
      \highlight[white]{\exists t^{\prime} \left(t \longrightarrow t^{\prime}\right)}
      \highlight[white]{\implies}
      \highlight{\vdash t^{\prime} {:} T}
    \right)
    $
  \end{overprint}

  \bigskip

  \begin{overprint}
    \onslide<+>
    $
    \highlight[white]{\text{For all}}
    \highlight[white]{\text{well-typed terms}}
    \text{,} \\
    \highlight[white]{\text{if}}
    \highlight[white]{\text{the term can take a step}} \\
    \highlight[white]{\text{the type is unchanged}}
    \text{.}
    $
    \onslide<+>
    $
    \highlight{\text{For all}}
    \highlight[white]{\text{well-typed terms}}
    \text{,} \\
    \highlight[white]{\text{if}}
    \highlight[white]{\text{the term can take a step}} \\
    \highlight[white]{\text{the type is unchanged}}
    \text{.}
    $
    \onslide<+>
    $
    \highlight[white]{\text{For all}}
    \highlight{\text{well-typed terms}}
    \text{,} \\
    \highlight[white]{\text{if}}
    \highlight[white]{\text{the term can take a step}} \\
    \highlight[white]{\text{the type is unchanged}}
    \text{.}
    $
    \onslide<+>
    $
    \highlight[white]{\text{For all}}
    \highlight[white]{\text{well-typed terms}}
    \text{,} \\
    \highlight{\text{if}}
    \highlight[white]{\text{the term can take a step}} \\
    \highlight[white]{\text{the type is unchanged}}
    \text{.}
    $
    \onslide<+>
    $
    \highlight[white]{\text{For all}}
    \highlight[white]{\text{well-typed terms}}
    \text{,} \\
    \highlight[white]{\text{if}}
    \highlight{\text{the term can take a step}} \\
    \highlight[white]{\text{the type is unchanged}}
    \text{.}
    $
    \onslide<+>
    $
    \highlight[white]{\text{For all}}
    \highlight[white]{\text{well-typed terms}}
    \text{,} \\
    \highlight[white]{\text{if}}
    \highlight[white]{\text{the term can take a step}} \\
    \highlight{\text{the type is unchanged}}
    \text{.}
    $
  \end{overprint}
\end{frame}

\begin{frame}[c]
  Progress and preservation give us type-safety: well-typed terms do not get stuck.
  \begin{itemize}
  \pause
  \item<+-> Progress means that well-typed terms are either values or can take a step.
  \item<+-> Values are not stuck, so if we are at a value we are done.
  \item<+-> Preservation means that well-typed terms that can take a step do not change type.
  \item<+-> After the step we have a well-typed term, so it is either a value or can
    take a step...
  \end{itemize}
\end{frame}

\begin{frame}[c]
  Progress and preservation also tie together syntax, semantics and typing.
\end{frame}

\begin{frame}[c]
  Together they mean we can use type systems to (approximately) classify which terms will or will not get stuck.
\end{frame}